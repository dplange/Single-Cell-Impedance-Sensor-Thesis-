\label{ch: discussion}

%\par The optimal device sensitivity changes with the sensor chamber height, width, cell size, and location.

%\par When chamber height is restricted, a wider electrode becomes helpful, if not negligible. When the electrodes widen more current  is allowed to flow through the higher regions of the chamber and include . However, this has diminishing returns

%\par The purpose of this thesis is to reevaluate and optimize the Cal Poly Biofluidic Lab's cell impedance spectroscopy system. To accomplish this, the following questions were answered:
Revisiting the thesis objectives set out in chapter \ref{ch: introduction}, the purpose of this thesis is to re-evaluate and optimize the Cal Poly Biofluidic Lab's cell impedance spectroscopy system by exploring the following questions:

\begin{enumerate}
    \item Are there aspects of the impedance spectroscope that need redesign or optimization?
    \item Is the hardware implementation correct?
    \item Can the correct circuit model and desired data acquisition and analysis be packaged and incorporated in a device interface computer program?
    \item Can model based design be applied to characterize and optimize electric fields and impedance to inform future designs?
    \item Can a device be manufactured at Cal Poly to validate the device design and analysis?
\end{enumerate}

\par To investigate these questions, the following was accomplished throughout the scope the thesis:
\begin{itemize}
    \item Fabrication of a micro-scale impedance spectroscopy chip.
    \item Design and implementation of a software interface to drive, measure, calculate, and compare impedance spectroscopy data in LabVIEW.
    \item Re-evaluation and validation of the impedance spectroscopy measurement circuit and requisite hardware.
    \item Evaluation of the impedance spectroscopy system by measuring impedance spectra of DI water, PBS, and a suspension of 7 $\mu$m polystyrene beads.
    \item Innovation of a novel volume fraction to fully account for cell geometry and fringe fields.
    \item Implementation of single-cell impedance spectroscopy models with numerical and analytic methods to investigate IS behaviour and optimization.
    \item Development of a graphical user interface for the analytic impedance solution to aid in model-based design.
\end{itemize}

\par To answer our guiding questions listed above, the findings from the aforementioned works are discussed and recommendations for future work are made.
    


\section{Micro-fabrication}

\par A fully-functional impedance spectroscopy chip was fabricated in the Cal Poly micro-fabrication lab. The microfluidic channels of the previous generation of the Biofluidic Lab's impedance spectroscopy chip was fabricated at Cal Poly, but the manufacturing of the micro-electrodes were outsourced to the UC Santa Barbara micro-fabrication lab. In this thesis, in addition to recreating the PDMS microfluidic channels, a process for micro-electrode fabrication was developed and demonstrated the capability to fabricate the complete impedance spectroscopy chip at Cal Poly.

\subsection*{Future Work for Device Manufacturing}

\par Although a complete BioMEMS impedance spectroscopy chip was successfully fabricated at Cal Poly, there is significant room for improvement to increase reliability and to decrease the scrap rate. There is a need for an alignment device to align the PDMS and the electrodes during glass bonding. The current solution is to align the device components by hand while observing the sensor region under a microscope. Making this alignment by hand is difficult and nearly impossible without adding ethanol to act as lubricant to provide about 10 minutes before bonding. It is possible that the rough process of hand alignment is responsible for some of the particulate generation that eventually led to the jamming and delamination of the IS chip. 

\par This thesis demonstrated the ability to create microelectrodes at the Cal Poly microfabrication lab; however, the process is fraught with risks that can destroy the electrodes and has a very high scrap rate. A huge improvement in the electrode manufacturing process could be realized by utilizing a dual target sputtering tool. Such a tool could allow the deposition of chromium and gold while keeping the substrate under vacuum. This is expected to prevent the formation of chromium oxide before the deposition of gold and significantly increase its adhesion.  


\section{IS Software}
\par To drive, measure, and calculate impedance spectroscopy experiments, a software interface was programmed in LabVIEW. The software is capable of making measurement from an I-V or auto-balancing bridge circuit; performing single IS experiments or timed IS runs at intervals of seconds to days; and features the the ability to compare previously recorded experiments. However, for the application of continuous IS measurements on the scale required for flow-through devices, there are significant updates required. 

\subsection*{Future Work}

\par If a flow-through impedance spectroscopy is developed, techniques to measure the broadband impedance spectra quickly as cells flow through the sensor region must be implemented. The currently implemented NI PXI-5421 function generator can output arbitrary waveforms can be programmed to apply a broadband signal and the frequency response can be analyzed with the Fast Fourier Transform, but the power distribution of the broadband signal, the short measurement time, and noise complicates the practical implementation. Several solutions to this problem have been proposed and include applications of chirp waveforms and maximum length sequences. \cite{min_broadband_2010, sun_digital_2009,sun_broadband_2007}. 

\par The current implementation of the IS software is built with the ability to incorporate new modules to allow for future features to be added to the program without affecting existing functions. These additions can be made by adding new states or state flows to the program. However, a large draw back to the current architecture is its limited ability of concurrency. If the need arises to make significant changes to the IS software, it is highly recommended to consider the implementation of a queued message handler architecture (QMH). The QMH architecture encourages parallel programming while also fortifying low-coupling practices with code modules that can co-exist and execute independently. It allows for a more responsive, faster, extendable, and maintainable program. Although likely overkill for this application, the object-oriented analogue to QMH, the ACTOR framework can also be considered. The actor framework ships standard with LabVIEW and its implementation can build a powerful architecture for medium to large applications.

\section{Impedance Measurement Hardware}
\label{sec:discussion on Impedance Measurement Hardware}

\par The impedance spectroscopy I-V measurement circuit was validated against a test circuit and a large portion of the reported error was quantified and used to improve upon the IS DAQ system. 

\par The measurement circuit was validated against a test circuit and its simulated response. The first limitation realized by evaluating the previous version of the IS hardware, is that the use of coaxial cables incurs significant errors and should be avoided in the current configuration. If coaxial cables are properly implemented, they are exceptional for radio frequency (RF) signal transmission featuring low loss, shielding to noise, and reduced RF emissions. However, when implemented with the I-V circuit described in this thesis, they are a source of significant error. Referring to figure \ref{fig:I-V_implementation}, if a coaxial cable is used to connect scope ch.1 to between the DUT and the external resistor, the characteristic capacitance of the coaxial cable would draw a significant amount of current through its shielding into ground and grossly invalidate the assumption that all current flowing through the DUT also flows through the external resistor. In addition, this implementation of the I-V circuit requires that the oscilloscopes are set to their high impedance settings rather than the 50 Ohm input used for avoiding reflectance with coaxial cable. However, if the auto-balancing bridge method is utilized, the benefits of coaxial cable can be obtained while the measurement circuit is protected from extra current draw by an operational amplifier, or equivalent, that maintains a virtual ground after the DUT. The current implementation utilizes insulated solid core wire while minimizing wire length to decrease noise susceptibility. 

\par After the implementation of solid core wire, effects of the oscilloscope on the recorded impedance was measured and quantified. Similar to the error caused by the coaxial cable, the current that leaks through the oscilloscope will artificially inflate the measured impedance. Errors were measured that reached over 10,000\%. This error was quantified as a factor of the external resistor and the effective external impedance and used to develop a correction factor. After application of the correction factor described in equation \ref{eqn:corrected_IV}, errors were reduced to under 30\% over the range of 3 kHz to 25 MHz. 

%\par As described in section ??? the impedance was validated against a test circuit and the majority of the error accounted for and quantified as current leaking through the oscilloscope connected between the DUT and the external resistor. An established range of ?? to ?? was found to result in low error after applying the correction function of equation ???.



\subsection*{Future Work}

\par All recorded IS datasets feature low-frequency noise that abates near 3 kHZ to 10 kHZ. This noise could be from small voltage signals approaching the noise floor, or the resolution limit of the oscilloscope. The cause of this noise was not pursued and is a shortcoming of the current IS measurement system that should be explored in future iterations.

\par Beyond the low-frequency noise, the broadband error can be further reduced by replacing breadboards for alternative connections that are less prone to parasitic capacitance, the implementation of high impedance FET oscilloscope probes, or the application of an autobalancing-bridge circuit. Practical auto-balancing bridge circuits can be complicated to implement accurately for broadband signals, but some have had success creating digital auto-balancing circuit systems \cite{li_high-speed_2013}.

\par For the current implementation, insulated solid core wire was connected to the impedance spectroscopy chip, and the length of the wire was minimized to reduce noise. A breadboard was used for making and organizing connections. Although no differences were noticed between data measured with and without the the use of a breadboard, future implementations should move away from using a breadboard due do its inherent capacitance.

\section{IS System Evaluation}

\par The complete IS system was evaluated on the basis of repeatability and qualitative accuracy through the comparison of measured impedance spectra of DI water, PBS, and 6$\mu$m polystyrene beads suspended in PBS. Due to debris flow that ultimately clogged the device, the evaluation was made on a limited body of data. However, the results were consistent and appeared sufficient to draw conclusions on the system's performance.

\par To assess the reproducibility of the IS system, impedance spectra of PBS measured over three consecutive days were compared. The data clearly showed that experiment could be reproduced over the broadband range of 10 kHz to 40 MHZ. However, the low frequency noise observed in the measurement hardware validation also occurred in IS chip measurements and caused high variability. Speculations on the source of this noise is discussed in  of DI water, PBS, and 6$\mu$m polystyrene beads suspended in PBS.

\par The accuracy was qualitatively assessed by comparing the relative impedance spectra of DI water, PBS, and a 6$\mu$m polystyrene bead suspension in PBS. The impedance spectra met the expectations set by the dielectric properties of the solutions. Deionized water exhibited the largest impedance due to its high resistance and the conductive PBS responded with significantly lower impedance. By saturating the sensor chamber with PBS suspended polystyrene beads, the capacitive impedance significantly increased over the PBS medium. 

%\par In this thesis we implemented an impedance spectroscopy system and measured the impedance spectra of DI water, PBS, and 6$\mu$m polystyrene beads suspended in PBS. Unfortunately, once the bead suspension was added to the device, debris began to flow and ultimately clogged the device and led to its delamination. However, enough data was captured to investigate the impedance spectroscopy reproducibility and performance. 

%\par The reproducibility studies analyzed in figure \ref{fig:IS_data_reproducibility} indicated that experimental data from 10 kHz to 40 MHz can largely be reproduced. Frequencies up to 10 kHz showed significant variation. However, this low frequency range persistently performed poorly across all experiments. Although tests were not performed to validate the idea, one possibility for the low frequency "noise" may be due to reaching the mininum voltage threshold set by the resolutino of the Oscilliscopes. The NI PXI-5124 12-bit resolution.

%\par The performance of the device was evaluated by measuring and comparing DI water, PBS, and 6$\mu$m polystyrene beads suspended in PBS. For all three test samples low quality data was recorded until around 10 kHz. However, the overall results met expectations based on the material properties of the samples and validated in large strokes the behaviour of the device.

\subsection*{Future Work}

\par Although experiments largely appeared promising and demonstrated a functioning impedance spectroscopy device, there is a significant amount of remaining work to bring a fully completed, reliable, and robust device on-line. First and foremost the manufacturing process should be revisited and refined with an emphasis on particulate reduction. Particles on the order of 10 $\mu$m will significantly reduce the life-span of the device. 

\par The investment in higher resolution photolithography masks will likely be required to manufacture single-cell impedance spectroscopy cell-capture devices. Alternatively, a flow-through device will be easier to fabricate with lower resolution masks. However, flow-through devices will require additional changes to the software and data post-processing if broadband frequencies need to be explored. In addition, the addition of a second set of electrodes to make differential measurements should be considered. The dual sets of electrodes will allow for real-time subtraction of the medium and EDL from the impedance spectra to isolate the effect of the cell. Alternatively, particle spectra isolation can be calculated from a single set of electrodes by comparing the the individually captured particle suspension spectra to the medium spectra as depicted in figure \ref{fig:diff}, however, this method runs the risk that the medium properties have changed between experiments.

\section{Impedance Spectroscopy Modeling}

\par To investigate and optimize the single cell impedance spectroscopy system an analytic and finite element model was developed. 

\par By following the work of Sun \cite{sun_analytical_2007}, an analytic solution of co-planar electrodes was developed through application of Maxwell's Mixture Theorem and conformal mapping. Part of this solution required the use of the volume fraction. Previously used volume fractions were calculated as the ratio of the cell volume to the volume over the electrodes, however, this method neglects the position of the cell and the non-uniformity of the electric field. To address this, the effective volume fraction based on conformal mapped volumes and dissipated power were developed. These methods increased the accuracy of the analytic impedance solution and allowed for the quantification of the effect of the cell position.

\par The analytic solution and the finite element model largely agreed with each other with percent errors reaching as low as 0.01\%. The finite element model further explored the effect of the device geometry and found that over 29\% of the electric current leaked through the overlap of the electrodes and flush channel. This current leak significantly decreases the device sensitivity. 


\par In section \ref{sec:optimization_results} device optimization was explored. With the ability to simulate the effect of position with novel effective volume fraction methods, it became obvious that there were two opposing qualities to optimize: the maximum sensitivity given a particle location and size, and the maximum sensitivity given a desired volume of uniform sensitivity. In both cases, the optimal geometry was highly dependent on the application. It was concluded that there is not optimal geometry for all scenarios. However, several trends were found that are critical for device design.

\par Given an electrode height of 10 $\mu$m, small to negligible gains in sensitivity were found with increased electrode width. This is an important finding of the optimization studies.  Although the sensitivity benefits are meager, wider electrodes are far easier to manufacture and are more resilient to cracks and defects. Physically, the modest effect of increased sensitivity is explained by opening additional parallel current paths with wider electrodes and, in effect, increasing the dissipated power in the region of the particle. However, if the particle is near the base of the sensor chamber and is small compared to the chamber height, increasing the electrode width will decrease the sensitivity since the majority of the additional dissipated power occurs above the particle. 

\par Since the electric double layer (EDL) can be modeled as a largely capacitive distributed impedance over the electrode surface, it may be tempting to conclude that by increasing the the width of electrodes, the impedance load of the EDL is reduced.  And while that is true, the effect rapidly diminishes. In general for a given channel height, additional electrode area distal to the sensor chamber center contributes less current than electrode area near to the sensor chamber. Likewise, the EDL on distal electrode areas contributes less capacitance to the system impedance. This can be visualized by revisiting our ideal capacitor model in the W-plane (\ref{fig:parallel_capacitor}). For each additional increase in electrode width, $\gamma$ becomes larger compared to $\delta$, but each additional increase in electrode width contributes progressively less to $\gamma$. 

\par The effect of channel height on sensitivity matched intuition. Smaller channel heights increase sensitivity in all scenarios by restricting parallel currents outside of the particle region and by focusing the particle to higher power regions. In addition, short channel heights enable wider electrodes. With increased channel heights, the distal portions of wider electrodes will contribute more current to the higher regions of the chamber and effectively reduce the sensitivity. With target particles of 6$\mu$m in diameter, a chamber height of 10$\mu$m in our current device near optimal. Although the sensitivity of the device could dramatically be increased with decreased channel height, it comes with the increased risk of clogs from particles and debris.

\par Optimizing the electrode gap to maximize the device sensitivity is highly dependent on the cell size, cell location, and channel height. The optimization for a 6$\mu$m  diameter cell centered in the sensor chamber with a channel heighth of 10 $\mu$m is found at an electrode gap of about 5 $\mu$m, which is already the electrode gap of our current device. However, if the cell deviates from this position, the sensitivity will rapidly diminish. If an accurate method of cell-capture is employed, this highly spatially-dependent sensitivity may be acceptable. However, for cell-capture devices where the cell position cannot be guaranteed or for flow-through devices, a large region of uniform power dissipation is desired. A uniform power region can be accomplished by increasing the electrode gap, but at the cost of high sensitivity. This can be useful for increasing the device resident time. 

%\par found that by increasingHowever,e the  the electrode width, small to negligible gains to the sensitivity were found. However, this i was one of the more important realizations of the optimization studies: there is little benefit to fabricating skinny electrodes. It is optimal to design wider electrodes. Although the benefits to the sensitivity is meager, wide electrodes are far easier to manufacture, are more resilient to non-contiguous causing defects and should decrease the overall scrap-rate. In addition, the wider electrodes will have a lower inherent impedance load. Another interesting benefit, is the effect of wider electrodes on the impedance load of the electric double layer. If the EDL behaves somewhat similar to a distributed capacitance over the fluid-contact electrodes, wider electrodes will work in our favor towards minimizing the effect of the EDL. It should also be noted that the channel height has a strong effect on the effect of electrode width. By increasing the channel height, the electrode extremities can start passing additional current that flows parallel to our cell by providing parallel current paths.

%\par Where wider better for electrode width, the wider electrode gap sensitivity maximization was highly dependent on the cell size, height in the channel, and the cell location. For maximizing the electrode gap for our current device with a 6 $\mu$m diameter cell assumed to be at half height, we are already maximized at a gap of 5 $\mu$m. However, figure \ref{fig:expanded_power_sensitivity} but you can begin to develop an area of uniform power dissipation that can be extremely useful in flow through designs where the latency??(retention) time is critical, or for removing the effect of cell position. 

%\par Although many prefer parallel facing electrodes, this study has demonstrated some power benefits of co-planar electrodes. To increase the sensor region for parallel electrodes, the width needs to be increased. By increasing the electrode width, the sensitivity drops. This is due to the opening of parallel current paths to bypass the particle. The sensitivity is dropping because the actual magnitude of cell impedance is dropping. If the impedance is dropping. IF the impedance difference was examined, it would be shrinking.

%\par To increase the measurement area with co-planar electrodes, the electrode gap is increased, and similarly the sensitivity would decrease. However, there are no parallel paths opening, rather there is just more medium to dissipate power and resist flow. As depicted in figure \ref{fig:expanded_impedance_difference}, the actual cell impedance due plateaus and does not change.

%\par In addition, by focusing cells closer to the electrode plane and decreasing the electrode gap, the sensitivity can be made to significantly surpass parallel electrodes.

\subsection*{Future Work}

\par In both the analytic and the finite element models the electric double layer is neglected. However, the electric double can easily shroud the effect of a single particle. Although techniques have been mentioned in this thesis to isolate the effect of the particle from the EDL, the development of an accurate EDL model would be useful for device validation and experiment interpretation. Current analytic models are largely inaccurate, but electrodes can be characterized by fitting experimental data to empirical models. 

%\par Although the effect of the electric double layer is effectively cancelled out, the large impedance load of the electric double layer can be a challenge in impedance spectroscopy may wish to be considered in device optimization. As discussed in section \ref{sec:the_electric_double_layer}, analytic solutions to the EDL can be inaccurate by a few orders of magnitude. However, it may be beneficial in considering the solution electrolyte concentration, the electrode material and surface finish, and the applied voltage among other parameters. 

%\par Since the EDL can be modeled as a distributed capacitance over the electrodes, it may be tempting to conclude that by increasing that the wider the electrodes the lower the EDL impedance in our system. And while that is true, it's effects rapidly decreases. This is because the effect of the EDL distributed capacitance is tied to the geometric constant ($G_f$) and more specifically $\gamma$ in the in w-plane. For any given height, the increase in effective capacitance due to the EDL from increases in the electrode width will taper off. If the channel height is increased, the EDL capacitance can be further increased by increased electrode width. And this makes sense, by increasing the channel height, additional parallel current paths open to further extremities of the electrode. However, as the channel becomes saturated with power at a specified height, increasing the electrode width. To decrease the impedance load of the EDL, the distributed capacitance must be increased. 

\section[IS Model-Based Design Framework]{A Framework for Single-Cell IS Model-Based Design}

\par As previously discussed, there is no optional design for all cases. To that end, a model-based design framework using the simulations and tools developed in this thesis is presented. The IS app presented in chapter \ref{ch: modeling} contains most of the tools required for design calculations in the following steps.

\begin{enumerate}
    \item \underline{Identify Functional Requirements and Material Properties}
    \par Prior to device design or fabrication, the functional requirements of the device and the relevant material properties should be identified. This information should include the device type (i.e. flow-through or cell-capture), the cell type, the literature values of the cell's dielectric properties, the average and variance of the cell size, the plausible cell mediums, and the dielectric properties of those mediums.
    
    %cells and their  will drive design in the following steps. The type of experiment to be run should be determined, where individually measuring a large quantity is more appropriate for flow-through devices and time-dependent studies of fewer cells is more appropriate for cell-capture devices. The type of cells to be measured and their required mediums should be identified along with their respective electric properties. The properties of the cell to be measured should identified.
    
    \item \underline{Determine Required Frequency Range}
    \par The required frequency range of the impedance spectroscopy system should be determined by the functional requirements identified in step 1. Cell counting or cell size can be determined with a DC or low-frequency applied signal. However, if information regarding the cell membrane or internals is required, a broadband frequency range will be required. A useful tool to determine the required frequency range is the Clausius Mossotti factor. By using the literature values of the dielectric properties of the cell and medium to be measured, theoretical values of the suspensions dielectric relaxations can be identified and used to inform a required frequency range.
    
    \item \underline{Model and Optimize Simple Device Design}
    \par The purpose of this step is to use the analytic IS solution to co-planar electrodes to inform initial design and electrode optimization. If the cell will be focused to a specific height, the electrodes can be optimized to the maximum sensitivity, but if the resident time needs to be increased, the electrodes can be optimized to create a region of uniform power density at the cost of maximum sensitivity. If the fluid dynamics requires a channel geometry that differs significantly from the simple model, a FEA simulation can quantify the device impedance and sensitivity.  However, it is still recommended to optimize the system with the system model as mesh changes throughout a parametric analysis can introduce “mesh noise” into the optimization curve. Once the general sensor region design is determined, an estimate of the impedance load of the system should be calculated (including an estimate of the EDL). 
    
    \item \underline{Design Impedance Spectroscopy Measurement Hardware}
    \par With an estimate of the required frequency range, the impedance load, and the device type, the impedance measurement system can be selected or designed. For measurement circuits, this thesis focused on the simplest I-V circuit and discussed the auto-balancing bridge method, but there are other methods that range in complexity and are appropriate for different frequency and impedance loads. See section \ref{sec:Circuit Implementation} for a short discussion of I-V and auto-balancing circuits and "A Guide to Measurement Technologies and Techniques" by Keysight Technologies for a more in-depth discussion of a wider range of measurement circuits \cite{keysight_technologies_impedance_2015}. The ability of the signal generator should be determined by the required frequency range and the type of the device. Depending on the type of the measurement circuit and impedance load, close attention should be made to the input impedance of the oscilloscope or the ADC and how this will affect measurement readings. 
\end{enumerate}