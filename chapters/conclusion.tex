
\par This thesis implemented a bioelectric impedance spectroscopy system and developed a model to inform design and aid in the interpretation of single-cell impedance spectroscopy.
\par To implement the impedance spectroscopy system, an impedance spectroscopy bioMEMS chip was fabricated in the Cal Poly Microfabcrication lab, software was developed to run impedance spectroscopy experiments, and impedance spectroscopy experiments were run to validate the system. 
\par To develop the single-cell impedance spectroscopy model, Maxwell's mixture theorem and the Schwartz-Christoffel transform was used to calculate an analytic impedance solution to the co-planar electrode system, a novel volume fraction to account for the non-uniformity of the electric field was developed to increase the accuracy of the analytic solution and to investigate the effect of cell position on the impedance spectrum, a software program was created to allow easy access to the analytic solution, and FEA models were developed to compare to the analytic solution and to investigate the effect of complex device geometry. This model was used to explore device optimization and to develop a model-based design framework.

\par This thesis found that the Cal Poly Biofluidics Lab's impedance spectroscopy system is a viable system to make cell measurements. With improvements to the IS measurement system and new IS chips guided by model-based design, this system can make repeatable, accurate, and meaningful single-cell impedance measurements.

%\par To lay a foundation to design, implement, and analyze the next generation of the Cal Poly Biofluidics Lab's single cell impedance spectroscope. 