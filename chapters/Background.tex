

%%%%%%%%%%%%%%%%%%%%%%%%%%%%%%%%%%
% Cells
%%%%%%%%%%%%%%%%%%%%%%%%%%%%%%%%%%
\section{Cells}
\par The cell is the basic unit of life. At the fundamental level, a cell must have an outer membrane that acts as a gatekeeper between the surrounding environment and the interior constituents. The contents of the cell interior varies widely between organisms, cell specializations, and states of each cell. For example, mature red blood cells do not contain a nuclei or any genetic material, but skeletal muscle fiber consist of multiple nuclei \cite{daniel_d_chiras_human_2005}. However, all these variations can be generalized to useful cell models, such as the model for human cells in figure \ref{fig:human_cell_model}.   
\begin{figure}[ht]
 \centering
 \includegraphics[width=\textwidth]{images/humanCellOverview.png}
 \caption[Diagram of human cell structure.]{Diagram of human cell structure \cite{daniel_d_chiras_human_2005}. }
 \label{fig:human_cell_model}
 \end{figure}
 
 \par Generally, a cell carries genetic information in the nucleus which resides inside the cell membrane and cytoplasm. Cytoplasm is the cell material inside of the cell membrane and outside the nuclei. The cytoplasm consists of non-nuclei organelles and cytosol.  Cytosol refers to the cellular solution between the cell organelles and the membrane, wich contains salts, nucleic acids and cytoskeleton filaments \cite{daniel_d_chiras_human_2005}. Organnelles are membrane bound structures inside the cell that perform special functions. Organelles include nuclei, mitochondria, the Golgi apparatus, lysosomes, and vacuoles.  
 
 %%%%%%%%%%%%%%%%%%%%%%%%%%%%%
 % The Cell Membrane
 %%%%%%%%%%%%%%%%%%%%%%%%%%%%%
 \subsection{The Cell Membrane}
 \par The cell membrane is an essential component of the cell that is necessary for signalling, extracellular interaction, and maintaining equilibrium. The membrane is primarily composed of the lipid bilayer, which is a semi-permeable barrier constructed from phospholipids. A phospholipid has a hydrophilic head group and a hydrophobic tail group. When exposed to an aqueous environment, phospholipids join together in a configuration that leaves the head groups exposed to the aqueous solution and the tail group surrounded by other non-polar tail groups. In the case of a cell, the phospholipids take the form of a spherical bilayer membrane (figure \ref{fig:cell_membrane}). The membrane forms a semi-permeable barrier that allows for restricted diffusion of certain molecules. 
 
 \par In addition to phospholipids, the cell membrane is packed with proteins that are critical for cell life and its interaction with the environment. Of interest to this thesis, are channel and transport proteins that allow controlled passage of molecules through the membrane.  
 
 \begin{figure}[h]
    \centering
    \includegraphics[width=\textwidth]{images/Cell_membrane_detailed_diagram.png}
    \caption[Diagram of the cell membrane]{Diagram of the cell membrane \cite{mariana_ruiz_cell_????}.}
    \label{fig:cell_membrane}
 \end{figure}
 
 
 %%%%%%%%%%%%%%%%%%%%%%%%%%%%%%%%%%%%%%%%%
 % Electrical Model of the Cell
 %%%%%%%%%%%%%%%%%%%%%%%%%%%%%%%%%%%%%%%%%
\subsection{The Electrical Model of the Cell}

\par The state of a cell directly affects its electrical properties. Measuring these properties provides insight into the cell state, type, and afflictions. A common representation of the bulk electrical model of a cell is presented in figure \ref{fig:electric_model_cell}. As a bulk electrical element, the cell membrane can be modeled as a leaky capacitor that can accumulate a charge differential, but also leaks charges across the membrane. This behaviour can be modeled as a resistor and capacitor in parallel. The membrane capacitance is supplemented by the formation of an electric double layer (EDL). A biological cell carries a net negative charge which results in the development of a surrounding 10-100 nm thick electric double layer in electrolyte solutions \cite{swaminathan_effect_2009}. The electric double layer contributes to the capacitance of the cell membrane. The EDL will be described in further detail in section \ref{sec:the_electric_double_layer}. The bulk electrical properties of the cytoplasm are often modeled as a RC series circuit with a relatively large capacitance. 

\begin{figure}[h]
    \centering
    \includegraphics[width=0.4\textwidth]{images/completeCellCircuit.png}
    \caption[Electrical model of single shelled cell.]{Electrical model of single shelled cell. $R'_i$ and $C'_i$ are the resistance and capacitance of the cytoplasm, and $R'_{mem}$ and $C'_{mem}$ are the resistance and capacitance of the membrane.}
    \label{fig:electric_model_cell}.
\end{figure}
 
 %%%%%%%%%%%%%%%%%%%%%%%%%%%%%%%%
 % Dielectric Spectroscopy
 %%%%%%%%%%%%%%%%%%%%%%%%%%%%%%%%
 \section{Dielectric Spectroscopy}

 % Read into articles and provide more details
 
 \par The dielectric properties of cells have been investigated since 1910 when H\"{o}ber showed the existence of the cell membrane by measuring the conductivity of erythrocytes at high and low frequencies \cite{hober_r_methode_1910}. The field of study further developed with Fricke's application of Maxwell's equations to measure the capacitance and thickness of the cell membrane from 1924 to 1925 \cite{james_clerk_maxwell_treatise_1892, fricke_h_mathematical_1924, fricke_h_electric_1924, fricke_h_electric_1931}. 
 
 \par In 1928, Cole used Maxwell's mixture equation to derive the complex impedance of a single shelled cell model and developed equations to describe the Cole-Cole plot \cite{cole_electric_1928}. And with Curtis, Cole made the first single cell measurements on a Nitella cell (a large bacteria cell that ranges from 20 to 60 mm in length) \cite{curtis_transverse_1937}. 
 
 \par From 1957-1968 Schwan used broadband electric impedance spectroscopy to identify $\alpha$, $\beta$, and $\gamma$ dispersions of a cell \cite{schwan_h_p_electrical_1957,schwan_h_p_electrical_1963,schwan_electrical_1994}. A dielectric dispersion is a frequency range with a large change in permittivity with respect to frequency. The permittivity of a material is the resistance to forming an electric field over a medium. In general the effects of a source charge can be described as the polarization of surrounding medium and the residual electric field over the resisting polarized medium. For a dielectric material, the electric field may be thought of as having two components: the electric displacement vector $\boldsymbol{D}$ that accounts for the electric field generated from separated free electrical charges, and the polarization $\boldsymbol{P}$ which accounts for the bound polarization charges in a dielectric material. The effect of $\boldsymbol{P}$ is to attenuate of the electric field. This relationship can be observed by rearranging the definition of electric displacement.
 \begin{equation}
     \boldsymbol{D} = \epsilon_o \boldsymbol{E} + \boldsymbol{P},
     \label{eqn:electric_displacement}
 \end{equation}
 \begin{equation}
    \boldsymbol{E} = \frac{1}{\epsilon_o}\Big(\boldsymbol{D} - \boldsymbol{P}\Big),
 \end{equation}
 
 \noindent where $\epsilon_o$ is known as the dielectric constant or the permittivity of the vacuum, with a value of 8.85E-12 F/m.
 
 \par For a linear dielectric medium, the polarization can be defined as 
 \begin{equation}
     \boldsymbol{P} = \epsilon_o\chi\boldsymbol{E},
     \label{eqn:polarization}
 \end{equation}
 \noindent where $\chi$ is the electric susceptibility. Substituting equation \ref{eqn:polarization} into \ref{eqn:electric_displacement}
 \begin{equation}
    \boldsymbol{D} = \epsilon_o(1+\chi)\boldsymbol{E} = \epsilon_o \epsilon_r \boldsymbol{E}=\epsilon\boldsymbol{E},
 \end{equation}
 
 \noindent where $\epsilon = \epsilon_o\epsilon_r$ is the permittivity and $\epsilon_r = 1 + \chi$ is the relative permittivity of a medium. For the case of isotropic materials, $\chi$ and $\epsilon$ reduce to scalar values, but for anisotropic materials, they are expressed as second order tensors. The susceptibility describes how a material polarizes under an electric field, and the permittivity describes the resistance to forming an electric field over that material. For dielectrics, the permittivity and susceptibility are closely related, but due to the resistance to forming an electric field over a vacuum, the two quantities are subtly different. The permittivity of a dielectric can be viewed as the sum of resistance to forming an electric field over a vacuum and the effect of polarizing a material.  Figure \ref{tab:permittivity_table} list the permittivity and susceptibility of some common materials.
 
 \begin{table}[h]
     \centering
     \includegraphics[width=0.4\textwidth]{images/permittivityTable.png}
     \caption[Permittivities of and susceptibilities of common materials]{Permittivities of and susceptibilities of common materials \cite{kirby_micro-and_2010}.}
     \label{tab:permittivity_table}
 \end{table}

 \par The frequency dependence of the permittivity arises from the time dependent material polarization or charge reorganization. Examples of charge reorganization include electron cloud distortion, charged particle displacement, and reorientation of molecules with dipoles. $\alpha$, $\beta$, and $\gamma$ dispersions refers to the ranges of alternating current frequencies where the permittivity decreases significantly in cells (i.e. dielectric relaxation). Figure \ref{fig:schwan_dispersions} depicts an approximate permittivity spectrum of a cell.
 
 \begin{figure}[ht]
 \centering
 \includegraphics[width=0.7\textwidth]{images/schwanDispersions.png}
 \caption[General cell permittivity spectrum]{General cell permittivity spectrum with labeled $\alpha$, $\beta$ and $\gamma$ dispersions.}
 \label{fig:schwan_dispersions}
 \end{figure}
 
 \par $\beta$ dispersions occurs up to 10 Mhz and is caused by the relaxation of the cell membrane. For lower frequencies, the membrane has enough time to charge and provide resistance to the electric field, but at frequencies greater than about 10 Mhz, the membrane does not have sufficient time to fully charge and provides little resistance to the electric field. $\gamma$ dispersion is related to the relaxation of water molecules. Instead of a charging membrane, $\gamma$ dispersion is due to the dipole rotation of the water molecules and the biomolecules that water is bound to. The source of $\alpha$ dispersion is undetermined, but current theories include the surface reactance of the electric double layer near the charged cell, the impedance of the sarcoplasmic recticulum (for muscle fibers), or the frequency dependent conductance of ionic membrane channels as predicted by the Hodgkin-Huxley equations \cite{schwan_electrical_1994}.
 
 
 \par These developments laid the foundations for the following techniques for measuring the dielectric properties of single cells. 
 
 %%%%%%%%%%%%%%%%%%%%%%%%%%%%%%%%
 % Dielectrophoresis
 %%%%%%%%%%%%%%%%%%%%%%%%%%%%%%%%
 \label{sec:dielectrophoresis}
 \subsection{Dielectrophoresis}
 \par The dielectrophoretic force (DEP) is the force generated on a particle suspended in a solution by the interaction of an applied nonuniform electric field and an induced dipole moment developed through Maxwell-Wagner polarization (a build up of charges at dielectric boundaries). Peter Debye and Herbert A. Pohl were among the first to develop dielectrophoresis and demonstrated how particles could be moved with nonuniform electric fields \cite{muller_potential_2003}. With an applied AC electric field, the average DEP force on a spherical particle is expressed as \cite{morgan_single_2007, green_dielectrophoresis_1999}
 
 \begin{equation}
     \big< F_{DEP} \big> = 2\pi \epsilon_m R^3 \text{Re}[\tilde{f}_{CM}] \nabla |\textbf{E}_\textbf{rms}|^2, 
     \label{eqn:dep_force}
 \end{equation}
 
 \noindent with
 
 \begin{equation}
     \tilde{f}_{CM} = \frac{\tilde{\epsilon}_p - \tilde{\epsilon}_m}{\tilde{\epsilon}_p + 2\tilde{\epsilon}_m}, 
     \label{eqn:fcm_background}
 \end{equation}
 
 \noindent where $\epsilon_m$ is the permittivity of the medium, $R$ is the radius of the particle, $\tilde{f}_{CM}$ is the Clausius-Mossotti factor for spherical particles,  $\tilde{\epsilon}_p$ is the complex permittivity of the particle, $\tilde{\epsilon}_m$ is the complex permittivity of the medium, and $\textbf{E}_\textbf{rms}$ is the root mean square of the electric field. For most cases, the complex permittivity can be expressed as 
 \begin{equation}
     \tilde{\epsilon} = \epsilon - j\frac{\sigma}{\omega}
 \end{equation}
\noindent where $\epsilon$ is the permittivity, $j = \sqrt{-1}$, $\sigma$ is the conductivity, and $\omega$ is the angular frequency of the electric field. 

\par From equation \ref{eqn:dep_force} and \ref{eqn:fcm_background}, it can be noted that, given a non-uniform electric field, the magnitude of the dielectrophoretic force depends on the size of the particle and the polarizability of the particle with respect to the medium (i.e Re$\big[\tilde{f}_{CM}\big]$), and the sign of the force only relates to the polarizability of the particle and medium (figure \ref{fig:freq_crossover}). It should be noted that, based on equation \ref{eqn:fcm_background}, the real part of the Clausius-Mossotti factor is restricted to less than 1 and greater than -0.5. However, for non-spherical shapes, such as elongated ellipses, equation \ref{eqn:fcm_background} does not satisfy the Clausius-Mossotti factor which can reach values far greater than 1.  


\begin{figure}[ht]
 \centering
 \includegraphics[width=\textwidth]{images/DEPCrossover.png}
 \caption[Frequency crossover of Clausius-Mossotti factor] {Frequency crossover of Clausius-Mossotti factor with labeled regions of positive and negative dielectriphoretic forces. From equation \ref{eqn:dep_force}, if $\tilde{\epsilon}_p$ is greater than $\tilde{\epsilon}_m$, then the DEP force is positive, if $\tilde{\epsilon}_m$ is less than $\tilde{\epsilon}_p$, then the DEP force is negative.}
 \label{fig:freq_crossover}
\end{figure}
 
 \par To extract dielectric properties of cells using dielectric forces, the DEP crossover frequency can be experimentally determined and compared to theoretical values \cite{morgan_single_2007}. The crossover frequency occurs when the real part of the Clausius-Mossotti factor equals zero and is where the DEP force switches from a positive to a negative DEP force (or \textit{vice versa}). Experimentally, this can be determined by the observation of a stationary cell in a non-uniform electric field. Theoretically, the crossover frequency can be described as
 
 \begin{equation}
     f_{cross} = \frac{1}{2 \pi}\sqrt{\frac{(\sigma_m-\sigma_p)(\sigma_p+2\sigma_m)}{(\epsilon_p-\epsilon_m)(\epsilon_p+2\epsilon_m)}},
     \label{eqn:f_cross}
 \end{equation}
  
 \noindent where $f_{cross}$ is the crossover frequency, subscript $p$ denotes the particle, and subscript $m$ denotes the medium. Equation \ref{eqn:f_cross} can be rewritten to include the Maxwell-Wagner relaxation frequency as \cite{morgan_single_2007}
 
 \begin{equation}
    f_{cross} = \frac{1}{2 \pi} \sqrt{\frac{\sigma_m - \sigma_p}{\epsilon_p - \epsilon_m}f_{MW}},
 \end{equation}
 
 \noindent with
\begin{equation}
    f_{MW} = \frac{1}{2\pi\tau_{MW}},
\end{equation}
\begin{equation}
    \tau_{MW} = \frac{\epsilon_p + 2\epsilon_m}{\sigma_p + 2\sigma_m},
\end{equation}
 
 \noindent where $f_{MW}$ and $\tau_{MW}$ are the Maxwell-Wagner relaxation frequency and time constant respectively. The Maxwell-Wagner relaxation frequency characterizes the frequency of $\beta$ dispersion described by Schwan \cite{morgan_single_2007}.
 
 \par For a single shelled model of a cell, the crossover frequency can be expressed as
 \begin{equation}
     f_{cross} = \frac{\sqrt{2}}{8\pi R C_{mem}}\sqrt{(4\sigma_m - RG_{mem})^2 - 9R^2G^2_{mem}},
 \end{equation}
 
 \noindent where $C_{mem}$ is the specific capacitance of the cell membrane and $G_{mem}$ is the specific conductance of the membrane.
 
 %%%%%%%%%%%%%%%%%%%%%%%%%%%%
 % Electrorotation
 %%%%%%%%%%%%%%%%%%%%%%%%%%%%
 \subsection{Electrorotation}
 
 \par Electrorotation is the applied torque to a polarized particle in a medium of different dielectric properties by a rotating electric field. Such a field can be created with a system of four electrodes with sinusoidal signals of 90\textdegree \;phase shifts between adjacent electrodes \cite{goater_electrorotation_1999}. The phenomenon of electrorotation was first documented by Pohl in 1978 and then developed into dependable methods by Arnold and Zimmerman in 1982 \cite{pohl_dielectrophoresis_1978-1, arnold_rotating-field-induced_1982}.
 
 \par The average torque exerted on a spherical particle can be expressed as \cite{morgan_single_2007}
 \begin{equation}
    \Tau_{\text{ROT}} = -4\pi \epsilon_m R^3 \text{Im}\big[\tilde{f}_{CM}\big]|\textbf{E}|^2,
    \label{eqn:ROT_torque}
 \end{equation}
 
 \noindent where $\Tau_{\text{ROT}}$ is the torque applied to the induced dipole of the particle, $\epsilon_m$ is the permittivity of the medium, $R$ is the radius of the spherical particle, $\textbf{E}$ is the electric field, and $\text{Im}\big[\tilde{f}_{CM}\big]$ is the imaginary part of the Clausius-Mossotti factor expressed in equation \ref{eqn:fcm_background}. Similar to the average dielectrophoretic force (equation \ref{eqn:dep_force}), the Clausius-Mossotti factor is the frequency dependent element and controls whether the particle rotates with or against the applied rotating electric field.
 
 \par The applied torque can be measured by the angular velocity of the particle, which can be expressed as \cite{morgan_ac_2003}
 \begin{equation}
     R_{\text{ROT}} = - \frac{\epsilon_m\text{Im}\big[\tilde{f}_{CM}\big]|\textbf{E}|^2}{2\eta}K, 
 \end{equation}
 \noindent where $R_{\text{ROT}}$ is the angular velocity, $\eta$ is the dynamic viscosity, and K is the scaling factor. When the angular velocity has reached equilibrium, the applied torque will be proportional $R_{\text{ROT}}$ and related by the scaling factor K. The electrorotation torque spectrum of a cell can be used, in tandem with equation \ref{eqn:ROT_torque}, to determine the dielectric properties of the cell. 
 
 \par By combining dielectrophoretic and electrorotation spectroscopy techniques, significant details of the dielectric properties can be obtained. Since dielectrophoretic spectroscopy determines the real Clausius-Mossotti factor spectrum, and electrorotation determines the imaginary Clausius-Mossotti factor, the full $\tilde{f}_{CM}$ spectrum can be deduced from which dielectric properties can be extracted. By separating the real and imaginary components of the Clausius-Mossotti factor, the following expressions are obtained \cite{hober_zweites_1912, morgan_single_2007}.
 \begin{equation}
     \text{Re}\big[\tilde{f}_{\text{CM}}\big] = \frac{(\sigma_p - \sigma_m)}{(1+\omega^2\tau_{\text{MW}}^2)(\sigma_p + 2\sigma_m)} + \frac{\omega^2\tau^2_{\text{MW}}(\epsilon_p-\epsilon_m)}{(1+\omega^2\tau_{\text{MW}}^2)(\epsilon_p+2\epsilon_m)},
 \end{equation}
 \begin{equation}
     \text{Im}\big[\tilde{f}_{\text{CM}} \big] = \frac{3\omega\tau_{\text{MW}}(\epsilon_p\sigma_m-\epsilon_m\sigma_p)}{(1+\omega^2\tau^2_{\text{MW}})(\epsilon_p+2\epsilon_m)(\sigma_p+2\sigma_m)}.
 \end{equation}
 \par A major shortcoming of the previously discussed techniques is the long time needed for measurements. An electrorotaion assay could potentially take up to several seconds per test and limit the number of cells tested. An alternative approach, electric impedance spectroscopy, allows for rapid dielectric spectroscopy, and with flow-through designs, allows dielectric spectroscopy of every cell in a sample. 
 
 %%%%%%%%%%%%%%%%%%%%%%%%%%%%%%%%%%%%%%%%%%%%%
 % Electrical Impedance Spectroscopy
 %%%%%%%%%%%%%%%%%%%%%%%%%%%%%%%%%%%%%%%%%%%%%
 \subsection{Impedance Spectroscopy}
 \label{sec:impedance_spectroscopy}
 \subsection*{Impedance}
 \par Electrical Impedance is defined as the opposition of a system to the flow of electrical current for an applied voltage. If a sinusoidal voltage is applied to an arbitrary system, then the voltage and current can be expressed as
 \begin{equation}
    V(w) = |V|\cos(wt-\theta_V),
    \label{eqn:V(w)}
 \end{equation}
 \begin{equation}
    I(w) = |I|\cos(wt - \theta_I),
    \label{eqn:I(w)}
 \end{equation}
 \noindent where $w$ is the angular frequency, $t$ is the time in seconds, and $\theta_V$ and $\theta_I$ are the phase of $V$ and $I$ respectively. Using Euler's formula, equations \ref{eqn:V(w)} and \ref{eqn:I(w)} can be recognized as the real part of the following:
 \begin{equation}
    \hat{V}(w) = |V|e^{j(wt-\theta_V)},
 \end{equation}
 \begin{equation}
     \hat{I}(w) = |I|e^{j(wt-\theta_I)}.
 \end{equation}
 
 \noindent Applying Ohm's law, the impedance of the system can be expressed as
 
 \begin{equation}
    Z = \frac{|V|}{|I|}e^{j(\theta_I-\theta_V)},
 \end{equation}
 \noindent or as 
 \begin{equation}
     Z = |Z|e^{j\theta},
     \label{eqn:impedance_euler}
 \end{equation}
 \noindent where $\theta$ is the difference in phase of the voltage from the current after the current passes through the system. Equation \ref{eqn:impedance_euler} sets the interpretation of impedance to be a vector that lives on the complex plane where $|Z|$ is the length of the vector, and $\theta$ is the angle of the vector from the positive real axis (figure \ref{fig:Impedance_diagram}).
 
 \begin{figure}[ht]
 \centering
 \includegraphics[width=0.7\textwidth]{images/impdedanceDiagram.png}
 \caption[Representation of impedance as a vector on the complex plane]{Representation of Impedance as a vector on the complex plane with a real component $R$ and imaginary component $X$. }
 \label{fig:Impedance_diagram}
 \end{figure}
 
 \par As a vector on the complex plane, impedance can be represented in rectangular coordinates and polar form:
 \begin{equation}
    Z = R + jX,
 \end{equation}
 \noindent where $R$ and $X$ are the real and imaginary components of the impedance vector, 
 \begin{equation}
     Z = |Z| \angle \theta,
     \label{eqn:z_polar}
 \end{equation}
 \noindent where $|Z|$ is the magnitude of the impedance vector and $\theta$ is the angle of the vector from the positive real axis. Given an AC applied signal, Ohm's law can then be expressed as
 \begin{equation}
    |V| = |Z||I|,
    \label{eqn:ohm_mag}
 \end{equation}
 \noindent and the phase of the voltage signal will be shifted from the current signal by $\theta$. 
 
 \begin{figure}[ht]
    \centering
    \includegraphics[width=0.9\textwidth]{images/ac_signal.png}
    \caption[Current Response of a system to an applied voltage]{Current Response of a system to a 1 volt applied potential. Using equations \ref{eqn:z_polar} and \ref{eqn:ohm_mag}, the impedance of the system can be expressed as $Z=\sqrt{2}\angle -\pi/4$.} 
    \label{fig:ac_signal}
 \end{figure}
 
 \par The rectangular form and polar form of the impedance are related as follows:
 \begin{equation}
     |Z| = \sqrt{R^2 + X^2},
 \end{equation}
 \begin{equation}
    \theta = \text{arctan}\Big(\frac{X}{R}\Big),
 \end{equation}
 \begin{equation}
    R = |Z|\cos\theta,
 \end{equation}
 \begin{equation}
     X = |Z|\sin\theta.
 \end{equation}
 
 \par The impedance of a system can be represented by a combination of elementary circuit components: resistors, capacitors, and inductors. The voltage response of these elements in the time domain can be expressed as
 \begin{alignat}{2}
    &\text{Resistor:} \quad  &&V = IR, \label{eqn:ohms_law}\\
    &\text{Capacitor:} \quad &&V = C \frac{dV}{dt},\\
    &\text{Inductor:} \quad  &&V = L \frac{dI}{dt},
 \end{alignat}
 \noindent where $V(t)$ and $I(t)$ are the voltage and current as a function of time, $R$ is the resistance, $C$ is the capacitance, and $L$ is the inductance. After applying Laplace transforms, the frequency dependent impedance of each element can be obtained as
  \begin{alignat}{2}
    &\text{Resistor:} \quad  &&Z(w) = R, \;\;\;\\
    &\text{Capacitor:} \quad &&Z(w) = \frac{1}{sC},\\
    &\text{Inductor:} \quad  &&Z(w) = sL,
 \end{alignat}
 
 \noindent where $s=jw$ for steady state solutions.

\subsection*{Impedance Spectroscopy: Coulter Counter}
\par One of the first devices capable of measuring the electrical properties of particles is the Coulter counter, which measures the direct current resistance of two fluid filled chambers connected by an aperture (figure \ref{fig:coulter_counter}). The apparatus drives fluid through the aperture by a pressure differential caused by different levels of fluid in the two chambers. As particles flow through the aperture, conductive fluid is displaced and there is a change in resistance. Pulses of increased resistance mark particles flowing through the aperture, and the magnitude of the resistance pulse relates to the size of the particle. 


\begin{figure}[ht]
    \centering
    \includegraphics[width=0.8\textwidth]{images/coultierCounter.png}
    \caption[Illustration of Coulter counter principles]{A Coulter counter that measures the DC resistance of the fluid in the chamber with two electrodes. When a particle enters the aperture, the change in resistance will provide information about the size and electrical properties of the particle}
    \label{fig:coulter_counter}
\end{figure}


In 1970, Deblois and Bean developed an analytical model of the Coulter counter resistance that can be related to dielectric properties of the particle \cite{deblois_counting_1970}. The resistance of the large chambers were assumed to be negligible and the resistance of the aperture was derived (figure \ref{fig:aperture}). For a tube of a mixture solution, the resistance can be expressed as

\begin{equation}
    R_t = \frac{4\rho_{mix}L}{\pi d_{t}} G,
    \label{eqn:tube_resistance}
\end{equation}

\noindent where $G$ is a correction term for the non-uniform current density, $R_t$ is the resistance of the tube, $L$ is the length of the tube, $d_{t}$ is the diameter of the tube, and $\rho_{mix}$ is the resistivity of the mixture, which can be expressed using Maxwell's approximation, 

\begin{equation}
    \rho_{mix} = \rho_m\bigg[ 1 + \frac{3\phi}{2} + \Big(\frac{-9\phi \rho_m}{4 \rho_p + 2 \rho_m}\Big)\bigg],
\end{equation}
\noindent where $\phi$ is the volume fraction, $\rho_m$ is the resistivity of the conductivity, and $\rho_p$ is the resistivity of the particle. This approximation is only valid with the assumption that the particle diameter is small relative to the aperture. The volume fraction of a spherical particle in a tube is

\begin{equation}
    \phi = \frac{2d_p^3}{3d^2_tL},
\end{equation}

\noindent where $d_p$ is the diameter of the particle.

\begin{figure}[ht]
    \centering
    \includegraphics[width=0.7\textwidth]{images/aperture.png}
    \caption[The aperture connecting the two chamber of the Coulter counter.]{The aperture connecting the two chamber of the Coulter counter where $L$ is the length of the aperture, $d_p$ is the diameter of the particle flowing through the aperture, and $d_{t}$ is the diameter of the aperture.}
    \label{fig:aperture}
\end{figure}
 
\par Since 2000, Coulter counters have been fabricated on the micro scale and have allowed increased sensitivity and adaptability, leading to nanopore devices capable of analyzing single molecules \cite{sun_single-cell_2010}.
 
 \subsection*{Impedance Spectroscopy: Microfluidic Impedance Cytometry}
 \par Instead of electrodes on either side of an aperture, as in Coulter counter designs, microfluidic impedance cytometry designs electrodes directly onto the walls of the test volume. The test volume is the region that contains the particle and fluid with significant current density. The impedance of the test volume can be approximated as a leaky capacitor:
 \begin{equation}
       \tilde{Z}_{mix} = \frac{1}{jw\tilde{C}_{mix}},
    \label{eqn:impedance_with_cap}
 \end{equation}
 
 \noindent where $\tilde{C}_{mix}$ is the complex capacitance of the test volume, and can be expressed as
 \begin{equation}
     \tilde{C}_{mix} = \tilde{\epsilon}_{mix}G_f,
 \end{equation}
 
 \noindent where $\tilde{\epsilon}_{mix}$ is the complex permittivity derived from Maxwell's mixture law and $G_f$ is the geometric factor that accounts for fringing electric fields. Further details are provided in section \ref{sec:theory_impedance_cytometry}.
 
 \begin{figure}[ht]
     \centering
     \includegraphics[width=\textwidth]{images/parallel.png}
     \caption{Parallel electrode configuration for impedance cytometry.}
     \label{fig:parallel_electrodes}
 \end{figure}
 
 \par Common electrode configurations are parallel and coplanar facing (figure \ref{fig:parallel_electrodes} and \ref{fig:coplanar_electrodes}). Coplanar electrodes are easier to fabricate than parallel facing electrodes, but come at the cost of diminished sensitivity and accuracy \cite{sun_analytical_2007}. Parallel electrode systems produce smaller variations in the electric field compared to coplanar electrodes. As a result, the signal variation is far greater for coplanar configurations. This results in a loss of accuracy, unless the vertical placement of the cell in the test volume can be controlled. In addition, parallel facing electrodes designs generally have larger volume fractions, and therefore, are more sensitive than coplanar electrodes. This can be observed by noting that given an electrode width, channel height, and channel depth, the coplanar test volume will be over twice that of parallel electrodes configurations.
  
 
 \begin{figure}[ht]
     \centering
     \includegraphics[width=\textwidth]{images/coplanar.png}
     \caption{Coplanar electrode configuration for impedance cytometry.}
     \label{fig:coplanar_electrodes}
 \end{figure}
 
 \par In microfluidic impedance cytometry, alternating currents at a series of frequencies are applied to the system to measure the impedance response. From this impedance spectrum, the frequency dependent response can be depicted as magnitude-phase plots, real-imamginary plots, and nyquist plots in order to gain insight into the behaviour of the system (figure \ref{fig:impedance_diagrams}). 
 
 \begin{figure}
    \centering
    \begin{subfigure}[b]{\textwidth}
        \centering
        \includegraphics[width=0.68\textwidth]{images/magPhaseExample.png}
        \caption{Magnitude-phase versus frequency diagrams}
        \label{fig:mag_phase_ex}
    \end{subfigure}
 
    \begin{subfigure}[b]{\textwidth}
        \centering
        \includegraphics[width=0.68\textwidth]{images/complexRealExample.png}
        \caption{Real-imaginary versus frequency diagrams}
        \label{fig:real_complex_ex}
    \end{subfigure}
    \begin{subfigure}[b]{\textwidth}
        \centering
        \includegraphics[width=0.52\textwidth]{images/nyquistExample.png}
        \caption{Nyquist plot}
        \label{fig:nyquist_plot}
    \end{subfigure}
    \caption[Impedance Diagrams]{Examples of magnitude-phase diagrams, real-complex diagrams, and a nyquist plot. The data is based off of the impedance of the circuit in figure \ref{fig:example_circuit}.}
    \label{fig:impedance_diagrams}
\end{figure}


\begin{figure}[t]
    \centering
    \includegraphics[width=0.3\textwidth]{images/exampleCircuit.png}
    \caption{Example circuit used in impedance diagrams.}
    \label{fig:example_circuit}
\end{figure} 
\newpage 

 \subsection*{Circuit Implementation}
 \label{sec:Circuit Implementation}

 \par The general process of impedance spectroscopy involves the application of a voltage or current signal, and the measurement of the signal response over a range of frequencies to obtain an impedance spectrum of the device under testing (DUT). The resulting spectrum provides insight into the dielectric properties of the DUT. 
 
 \par The simplest circuit for measuring impedance is based on the I-V method (figure \ref{fig:IV_impedance_measurement}). The I-V method determines the impedance of a DUT by measuring the voltage drop over the DUT and a high precision resistor in series. The current through the system can be calculated with Ohm's law at the high precision resistor. The impedance of the DUT for the I-V method can be expressed as 
 
 \begin{equation}
     Z_{DUT} = \frac{\Delta V_{DUT}}{I} = \frac{V_1 - V_2}{V_2}R,
     \label{eqn:IV_Z}
 \end{equation}
 
 \noindent where $V_1$, $V_2$, and $R$ correspond to values of elements in figure \ref{fig:IV_impedance_measurement}. An important disadvantage of this configuration, is that the accuracy of the impedance values relies on the precision of the resistor value, the extent of the parasitic capacitance in the system, and quality of the probes measuring $V1$ and $V2$ which can be challenged under a large DUT impedance and high frequencies. If the I-V method is used, coaxial cables will result in inaccuracies and should not be used.
 
 \begin{table}[ht]
    \centering
    \includegraphics[width=\textwidth]{images/impedanceMeasurementMethods.png}
    \caption[Common impedance measurement methods.]{Common impedance measurement methods. \cite{keysight_technologies_impedance_2015}}
    \label{tab:z_measurement_methods}
 \end{table}
 
 \begin{figure}[ht]
    \centering
    \includegraphics[width=0.7\textwidth]{images/I-VMethod.png}
    \caption[I-V impedance measurement configuration.]{I-V impedance measurement.}
    \label{fig:IV_impedance_measurement}
\end{figure}

\par Another circuit configuration to measure impedance is the auto-balancing bridge method (figure \ref{fig:auto-balancing_bridge}) \cite{keysight_technologies_impedance_2015}. The auto-balancing bridge method works as an extension of the I-V method. Instead of measuring the current directly, the potential on the low side of the DUT is driven to a virtual ground by an op-amp, and an equivalent current is measured. By applying Kirchoff's current law to the virtual ground node, the impedance of the DUT for using the auto-balancing bridge method can be expressed as

\begin{equation}
    \frac{V_1}{Z_{DUT}} - \frac{V_2}{R} = 0,
\end{equation}
\begin{equation}
    Z_{DUT} = \frac{V_1}{V_2}R.
\end{equation}

A significant advantage of the auto-balancing bridge method, is that when used with coaxial-grounded cables, the parasitic capacitance is removed from the measurement since all capacitances and measured elements are connected to ground. 
\clearpage

 \begin{figure}[ht]
    \centering
    \includegraphics[width=\textwidth]{images/autoBalancingBridge.png}
    \caption[Auto-balancing bridge impedance method.]{Auto-balancing bridge impedance method.}
    \label{fig:auto-balancing_bridge}
\end{figure}
 
 % Electrical impedance spectroscopy has historically only been used on multiple cells to obtain aggregate data, however, with the rise of microelectomechanical systems (MEMS) and microfluidics, electrical impedance spectroscopy can be applied to individualy cells.
 
 %%%%%%%%%%%%%%%%%%%%%%%%%%%%%%%%%%%%
 % Theoretical Model of Impedance
 %%%%%%%%%%%%%%%%%%%%%%%%%%%%%%%%%%%%
 \section[Model of Cell Suspension Impedance]{Theoretical Model of Impedance Cytometry}
 \label{sec:theory_impedance_cytometry}

%%%%%%%%%%%%%%%%%%%%%%%%%%%%%%%%%%%%%
% Maxwell's Mixture Theory
%%%%%%%%%%%%%%%%%%%%%%%%%%%%%%%%%%%%%
\subsection{Maxwell's Mixture Theory}
\label{sec:maxwell_mixture_theory}
 \begin{figure}[ht]
 \centering
 \includegraphics[width=0.7\textwidth]{images/cellAndElectrodes.png}
 \caption[Schematic diagram of simplified impedance sensor chamber.]{Diagram of the simplified impedance sensor chamber where w, $g$, and $l$ are the width, gap, and length of the electrodes respectively, and $h$ is the height of the chamber.}
 \label{fig:simplified_IS}
 \end{figure}
 
  \par The impedance of a single cell suspension, such as depicted in figure \ref{fig:simplified_IS}, can be solved for with Maxwell's mixture thoery \cite{james_clerk_maxwell_treatise_1892, sun_single-cell_2010}. Maxwell's equation for the complex permittivity of a mixture is
  \begin{equation}
      \tilde{\epsilon}_{mix} = \tilde{\epsilon}_m\frac{1 + 2\Phi\tilde{f}_{CM}}{1-\Phi\tilde{f}_{CM}},
  \end{equation}
  
  \noindent where $\tilde{\epsilon}_{mix}$ and $\tilde{\epsilon}_m$ are the complex permittivity of the mixture and medium respectively, $\tilde{f}_{CM}$ is the Clausius Mossotti factor, and $\Phi$ is the volume fraction. For most cases, the complex permittivity can be expressed as
  \begin{equation}
    \tilde{\epsilon} = \epsilon - j\frac{\sigma}{\omega},
\end{equation}

\noindent where $\epsilon$ is the permittivity, $j = \sqrt{-1}$, $\sigma$ is the conductivity, and $\omega$ is the angular frequency. The Clausius Mossotti factor is defined as
  \begin{equation}
    \tilde{f}_{CM} = \frac{\tilde{\epsilon}_p - \tilde{\epsilon}_m}{\tilde{\epsilon}_p + 2\tilde{\epsilon}_m}, 
  \end{equation}
  
  \noindent where $\tilde{\epsilon}_p$ is the complex permittivity of the particle.

 \begin{figure}[ht]
 \centering
 \includegraphics[width=0.7\textwidth]{images/singleShelledCell.png}
 \caption[Diagram of single shelled cell model.]{A diagram of a single shelled cell model where $\sigma$ is the conductivity, $\epsilon$ is the permittivity, and the subscripts $m$ and $i$ denote the membrane and cytoplasm respectively. $R$ and $d$ are the radius of the cell and membrane thickness respectively.}
 \label{fig:single_shell}
 \end{figure}

  \par The permittivity of the single-shelled cell in figure \ref{fig:single_shell}, can be modelled as
  \begin{equation}
      \tilde{\epsilon}_p = \tilde{\epsilon}_{mem} 
      \frac{\gamma^3+2(\frac{\tilde{\epsilon}_i - \tilde{\epsilon}_{mem}}
      {\tilde{\epsilon}_i + 2\tilde{\epsilon}_{mem}})}{\gamma^3 - (\frac{\tilde{\epsilon}_i - \tilde{\epsilon}_{mem}}{\tilde{\epsilon}_i + 2\tilde{\epsilon}_{mem}})} \;\;\text{  with } 
      \gamma = \frac{R + d}{R}, 
  \end{equation}
  
  \noindent where $\tilde{\epsilon}_i$ is the complex permittivity of the cytoplasm, $\tilde{\epsilon}_{mem}$ is the complex permittivity of the cell membrane, $R$ is the radius of the cell, and $d$ is the thickness of the cell membrane.
  
%  \par A corrected volume fraction is used to compensate for the fringe fields of the non-uniform electric field produced by the electrodes in figure \ref{fig:simplified_IS} \cite{gawad_micromachined_2001}. 
  
 % \begin{equation}
 %     \Phi = \frac{4}{3} \pi R^3 \frac{1}{G_f\text{w}h},
 %     \label{eqn:corrected_vf}
 % \end{equation}
  
 % \noindent where $h$ is the height of the channel, w is the width of the electrode and $G_f$ is a geometric factor dependent on the geometry of the channel and electrodes.
  
  \par The impedance of the mixture is
  \begin{equation}
    \tilde{Z}_{mix} = \frac{1}{jw\tilde{C}_{mix}},
    \label{eqn:impedance_with_cap}
  \end{equation}
  
  \noindent where $w$ is the angular frequency and $\tilde{C}_{mix}$ is the complex capacitance of the mixture which can be expressed as
  \begin{equation}
      \tilde{C}_{mix} = \tilde{\epsilon}_{mix} G_f.
      \label{eqn:capacitance_mix}
  \end{equation}
  
  \noindent If equations \ref{eqn:impedance_with_cap} and \ref{eqn:capacitance_mix} are combined, we obtain
  \begin{equation}
    \tilde{Z}_{mix} = \frac{1}{jw\tilde{\epsilon}_{mix}G_f}.
    \label{eqn:impedance_with_Gf}
  \end{equation}
  
  \par To find the value of the geometric factor $G_f$, the cell constant of electrodes must be determined.
  
  
  %%%%%%%%%%%%%%%%%%%%%%%%%%%%%%%%%%%
  % Electrode Cell Constant
  %%%%%%%%%%%%%%%%%%%%%%%%%%%%%%%%%%%
  \subsection{Electrode Cell Constant}
  \label{sec:electrode_cell_constant}
  \par The cell constant $\kappa$ is defined as the proportionality factor between the measured resistance $R_b$ and the resistivity $\rho$ of a material:
  \begin{equation}
      R_b = \kappa \rho.
      \label{eqn:cell_constant_proportionality}
  \end{equation}
  
  \noindent  Olthius related the measured resistance to capacitance in order to derive an analytic solution to the cell constant \cite{olthuis_theoretical_1995}. 
  
  \par To find $R_b$ for two electrodes with an interspatial material, the measured resistance can be related to capacitance via Ohm's law and Maxwell's equation:
  \begin{equation}
      RC = \frac{\oiint \epsilon \boldsymbol{E} \cdot d\boldsymbol{S}}{\oiint \sigma\boldsymbol{E}\cdot d\boldsymbol{S}},
      \label{eqn:RC_integral}
  \end{equation}
  
  \noindent where $R$ and $C$ are the resistance and capacitance between the electrodes, $\epsilon$ is the product of the relative and vacuum permittivity, $\boldsymbol{E}$ is the electric field vector, and the integral is taken over a surface including one electrode.
  
  \par If the interspatial material is homogeneous and isotropic, then equation \ref{eqn:RC_integral} can be reduced to
  \begin{equation}
      RC = \frac{\epsilon}{\sigma}.
      \label{eqn:RC}
  \end{equation}
  
  \par If we take $R_b = R$, we can combine equation \ref{eqn:RC} and \ref{eqn:cell_constant_proportionality} to express the cell constant in terms of capacitance:
  \begin{equation}
      \kappa = \frac{\epsilon}{C}.
      \label{eqn:cell_constant_C}
  \end{equation}
  
   \begin{figure}[ht]
  \centering
  \includegraphics[width=0.7\textwidth]{images/capacitorNoFringe.png}
  \caption[Uniform electric field between parallel plates]{Uniform electric field between two parallel electrodes where $\boldsymbol{E}$ is the electric field, $\phi$ is the voltage, and $\frac{d\phi}{d\boldsymbol{n}}=0$ is the boundary condition of a perfect insulator. The dimensions of the capacitor are the electrode height $\gamma$, and the distance between the electrodes $d$.}
  \label{fig:parallel_capacitor}
  \end{figure}
  
  \par The capacitance of the two parallel plates with a uniform electrode field in figure \ref{fig:parallel_capacitor} is
  \begin{equation}
      C = \frac{\epsilon A}{d},
      \label{eqn:capacitor}
  \end{equation}
  
  \noindent where $A$ the area of the plate, and $d$ is the distance between the plates. Since $A = l\gamma$, where $l$ is the width, and $\gamma$ is the height of the electrode, the capacitance per unit width can be written as
  \begin{equation}
      C_l = \frac{\epsilon\gamma}{d} \;\;\;\text{where} \;\; C =l\, C_l.
      \label{eqn:specific_capacitance}
  \end{equation}
  
  \par Returning to equation \ref{eqn:cell_constant_C}, and substituting equation \ref{eqn:specific_capacitance}, the cell constant can be expressed as 
  \begin{equation}
      \kappa = \frac{d}{\gamma \, l}.
      \label{eqn:cell_constant}
  \end{equation}
  
  \noindent The geometric factor is the inverse of the cell constant
  \begin{equation}
       G_f = (\kappa)^{-1},
       \label{geometric_cell}
  \end{equation}

  \noindent and can be expressed as
  \begin{equation}
      G_f = \frac{\gamma l}{d}.
      \label{eqn:geometric_constant}
  \end{equation}
  
  \par Equation \ref{eqn:geometric_constant} is the solution of the geometric constant for the electrode configuration in figure \ref{fig:parallel_capacitor}, but to find the geometric constant for any other configuration, including the coplanar electrode in figure \ref{fig:simplified_IS}, $d$ and $\gamma$ will need to be mapped to the other configuration. This can be accomplished with conformal transformations.
  
  
  %%%%%%%%%%%%%%%%%%%%%%%%%
  % Conformal Mapping
  %%%%%%%%%%%%%%%%%%%%%%%%%
  \subsection{Conformal Transformations}
  
  \begin{figure}[h]
  \centering
  \includegraphics[width=\textwidth]{images/mapping.png}
  \caption[Illustration of complex mapping.]{An Illustration of complex mapping.}
  \label{fig:mapping}
  \end{figure}
  
  \par Let $z = x + j\,y$, where $j$ is the imaginary number $\sqrt{-1}$, then a function of $z$, such as $W(z) = u(x,y) + j\,v(x,y)$, can be considered a mapping of an area of one complex plane to an area in another complex plane (figure \ref{fig:mapping}). Conformal transformations are a special kind of mapping between two complex planes that preserves local angles. A mapping is conformal if it is composed of analytic functions, and as a consequence, fulfills the Cauchy-Rieman equations. Conformal mappings are extremely useful for solving complicated problems by mapping the problem to a simplified domain. An example of a conformal mapping is $w(z) = tan(z)$, which maps an infinite vertical strip to a circle (figure \ref{fig:circleMapping}). 

   \begin{figure}[h]
    \centering
    \begin{subfigure}[b]{0.45\textwidth}
        \centering
        \includegraphics[width=\textwidth]{images/tanMappingStrip.png}
        \caption{Part of the partial infinite strip $-\pi/4<x<\pi/4$.}
    \end{subfigure}
    \hfill
    \begin{subfigure}[b]{0.45\textwidth}
        \centering
        \includegraphics[width=\textwidth]{images/tanMapping.png}
        \caption{Mapping of the partial infinite vertical strip to a circle }
    \end{subfigure} 
    \caption[Conformal mapping of vertical strip to circle]{An example of conformal mapping by transforming a partial infinite vertical strip to a circle with the mapping $w(z) = tan(z)$.}
    \label{fig:circleMapping}
 \end{figure}
  
 
    \par Sun, Greene, et al. utilized the Schwartz-Christoffel transform to map the coplanar electrode configuration in figure \ref{fig:simplified_IS} to the configuration of parallel electrodes with uniform electrode fields in figure \ref{fig:parallel_capacitor} \cite{sun_analytical_2007}. The Schwartz-Christoffel formula is a powerful transform that allows the mapping of the upper complex T-plane ($y>0$) to the inside of a polygon. The formula is
    
    \begin{equation}
        Z = C_1 \int_{T_0}^T \prod^m_{r=1} (T - T_r)^{(\theta_r/\pi - 1)} dT + C_2,
    \end{equation}
    
    \noindent where $Z$ is the interior of a polygon in the Z-plane with vertices $Z_1,\;Z_2,\;Z_3,\; ...,Z_m$ and angles $\theta_1,\;\theta_2,\;\theta_3,\; ...,\theta_m$ which correspond to the points $T_1,\;T_2,\;T_3,\; ...,T_m$ on the real axis of the T-plane. $C_1$ and $C_2$ are integration constants. The Schwartz-Christoffel transform has three degrees of freedom, and consequently, up to three points may be chosen arbitrarily. $T_0$ is the reference and is typically chosen at the origin.
    
    \par The solution and application of the Schwartz-Christoffel transform to co-planar electrodes will be further explored in chapter \ref{ch: modeling}.
    
    \FloatBarrier
    
  %  \par To find the geometric constant for coplanar electrodes, Schwartz-Christoffel transforms will be used to map the coplanar electrode geometry (Z-plane) to the upper complex plane (T-plane), and then to map the T-plane to the simplified W-plane (configuration of figure \ref{fig:parallel_capacitor}). The W-plane vastly simplifies the solution to the cell constant and the goemetric constant can be solved for with equation \ref{eqn:geometric_constant}. 
    
 
    
 \subsection{Cell Suspension Equivalent Circuit}
 \label{sec: cell_suspension_equiv_circ}
 \par With a solution to the impedance of the system, an equivalent circuit of the cell and medium can be used to approximate discrete components. A simple equivalent circuit was described by Foster and Schwan that describes the cell as a capacitor and resistor in series (figure \ref{fig:simple_equiv_circuit_cell_medium}) \cite{schwan_electrical_1994}. The model assumes that the resistance of the cell membrane is far greater than its reactance and can be expressed as a capacitor. Similarly, it is assumed that the cytoplasm resistance is far greater than its capacitance and can be modeled as a resistor.
 
 \begin{figure}
     \centering
     \includegraphics[width=0.5\textwidth]{images/simpleCellMediumCircuit.png}
     \caption{Simple equivalent circuit of cell and medium.}
     \label{fig:simple_equiv_circuit_cell_medium}
 \end{figure}
 
 \noindent The values of the simplified circuit model are as follow \cite{morgan_single_2007}:
 \begin{equation}
     R_m = \frac{1}{\sigma_m(1-3\Phi/2)G_f},
 \end{equation}
 \begin{equation}
     C_m = \epsilon_\infty G_f,
 \end{equation}
 \begin{equation}
     C_{mem} = \frac{9\Phi RC_{mem,0}}{4}G_f,
 \end{equation}
 \begin{equation}
     R_i = \frac{4\Big(\frac{1}{2\sigma_m}+\frac{1}{\sigma_i}\Big)}{9\Phi G_f},
 \end{equation}
 
 \noindent where $\sigma_m$ and $\sigma_i$ are the conductivities of the medium and cytoplasm respectively, $R$ is the cell radius, $\Phi$ is the volume fraction, and $C_{mem,0}$ is the specific membrane capacitance and can be expressed as \cite{sun_single-cell_2010}
 \begin{equation}
   C_{mem,0} = \epsilon_{mem}/d,
 \end{equation}
 
 \noindent where $\epsilon_{mem}$ is the permittivity of the cell membrane and d is the thickness of the membrane. $\epsilon_\infty$ is the high frequency permittivity of the suspension and can be approximated as
 \begin{equation}
     \epsilon_\infty \approx \epsilon_m \bigg[1-3\Phi\frac{\epsilon_m-\epsilon_i}{2\epsilon_m+\epsilon_i}\bigg].
 \end{equation}
 
 \par For cases where the cytoplasm reactance and membrane resistance are significant, such as during cell lysis, the complete equivalent circuit can be used (\ref{fig:complete_equiv_circuit_cell_medium}) \cite{sun_single-cell_2010}. A quantitative description of the model is described by Sun et al. \cite{sun_dielectric_2007}.

 \begin{figure}
     \centering
     \includegraphics[width=0.5\textwidth]{images/completeCellMediumCircuit.png}
     \caption{Complete equivalent circuit of cell and medium.}
     \label{fig:complete_equiv_circuit_cell_medium}
 \end{figure}
 
 %%%%%%%%%%%%%%%%%%%%%%%%%%%%%%%%%%%%%%%%%%%%%%%%%%%%%%%%%
 % Microelectromechanical Systems and Microfluidics
 %%%%%%%%%%%%%%%%%%%%%%%%%%%%%%%%%%%%%%%%%%%%%%%%%%%%%%%%%
 \section[MEMs and Microfluidics]{MEMS and Microfluidics}
 
 \par Microelectromechanical systems (MEMS) are devices on the order of microns, and are often smaller than the diameter of a human hair (about 100 microns). MEMS evolved from the manufacturing processes used to fabricate integrated circuits and ink jet cartridges \cite{xia_soft_1998-1}. Using silicon based microfabrication techniques such as photolithography, etching, and metal deposition; mechanical and electrically driven micro-sized pumps, cantilevers, and sensors can be fabricated \cite{wang_bio-mems:_2006}. 
 
 \par Using these microfabrication techniques, systems can be miniturized in precise and reproduceable packages with applications in a variety of biology related problems. MEMS applied in these fields are referred to as BioMEMS. The majority of BioMEMS are focused on creating diagnostic systems that can identify diseases and properties of biogical substances. These devices usually need to treat, filter, and utilize detection methods that include electrophoresis, dielectrophoresis, surface plasmon resonance, and enzyme-linked immunosorbent assays (ELISA) \cite{foudeh_microfluidic_2012}. In order to run diagnostics on a solution of biological material, a sample needs to undergo pretreatment, sample preparation, preconcentration, and detection. With the miniaturization afforded by microfabrication technology, the concept of combining these steps into a single chip is now feasible. These devices are referred to as labs on a chip or micro total analysis systems. This thesis focuses on the detection method of such a device.
 
 %%%%%%%%%%%%%%%%%%%
 % MEMS Fabrication
 %%%%%%%%%%%%%%%%%%%
 \subsection{MEMS Fabrication}
 
 \subsection*{Photolithography}
 
 \par Photolithography is a MEMS fabrication method that patterns a thin layer of photoresist onto a substrate. Photoresist is a polymer that reacts to certain wavelengths of light. How the photoresist reacts determines whether the photoresist is classified as a positive or negative photoresist. Positive photoresist becomes soluble to a developer after exposure to light, and negative photoresist becomes cross-linked and insoluble to the developer after exposure. 
 
 \begin{figure}[h]
     \centering
     \includegraphics[width=\textwidth]{images/negative_vs_positive_resist.png}
     \caption[A comparison of positive and negative photoresist]{A comparison of positive and negative photoresist. For positive photoresist, exposed regions are removed, and for negative photoresist, unexposed regions are removed.}
     \label{fig:negative_vs_positive_resist}
 \end{figure}
 
 \par Photoresist can be applied to a substrate with a precise thickness through application of a spinner machine. The machine spins the substrate at given rotation velocities to evenly distribute the photoresist and precisely control the thickness. 
 
 \par The photoresist can then be exposed to ultra violet light. In order to apply a design to the surface, a contact aligner can be used to expose the substrate under a mask. The mask is a transparency sheet with the design printed on it. The mask controls where light is allowed and the contact aligner controls the distance between the mask and the photoresist. The resist is then developed to reveal the patterned design. 
 
 \subsection*{Soft-lithography}
 
 \par Soft-lithography uses photolithography to create a mold of the desired structures. A material such as polydimethyl siloxane (PDMS) can be cast on the mold and then removed to create the desired structure. The PDMS can be plasma bonded to a substrate to create fluid channels. Soft-lithography is frequently used for the fabrication of microfluidic channels since the process is far easier and cheaper than the alternative of glass etching \cite{whitesides_soft_2001}.
 

 
 \subsection*{Metal Deposition}
 
 \par One of several methods for MEMS metal deposition is sputtering. Sputtering ejects metal ions onto the target surface. The target substrate is placed on the anode with the desired metal used as the cathode. An argon plasma bombards the metal plate and ejects metal ions that are attracted to the anode and deposit on the target substrate. 
 

 
  \subsection*{Lift-off Process}
 \label{sec:lift-off}
 \par The lift-off process is a method for precise metal deposition of a pattern onto a substrate and is useful for creating micro-scale electrodes. Photolithography is first used to pattern photoresist where metal is not desired. After metal deposition of the substrate with patterned photoresist, the sputtered substrate is then placed in photoresist remover to lift off the photoresist with the undesired metal.

 
 \subsection*{Plasma Bonding}
 
 \par To irreversibly bond PDMS, the surface can be exposed to an air plasma. The plasma activates the PDMS surface by creating Si-OH groups that can bond to itself, glass, silicon, polystyrene, polyethylene, and silicon nitride \cite{mcdonald_polydimethylsiloxane_2002-1}.


 

 
 
 
 %%%%%%%%%%%%%%%%%%%%%%%%%%
 % Microfluidics
 %%%%%%%%%%%%%%%%%%%%%%%%%%
 \subsection{Microfluidics}
 
\par In most BioMEMS there is fluid flow on the micrometer scale. At this scale fluid flow physics differ from fluid flow at the macro scale. Understanding and leveraging microfluidic mechanics allows for small reagent and sample volumes, multiplexing, and physic phenomenon that allow experiments and functions not possible at the macro scale. Laminar flow, diffusion, fluidic resistance, surface area to volume, and surface tension, may not be dominant in phenomenon on the macro scale, but on the micro scale become significant \cite{david_j._beebe_physics_2002}. Microfluidics are discussed in further detail in appendix \ref{app:microfluidics}.


%%%%%%%%%%%%%%%%%%%%%%%%%%%%%%%%%%%%%%%%%%%%%%%%%%%%%%%%%%%%%%%%%%%%
% The Electric Double Layer
%%%%%%%%%%%%%%%%%%%%%%%%%%%%%%%%%%%%%%%%%%%%%%%%%%%%%%%%%%%%%%%%%%%%
\section{The Electric Double Layer}
\label{sec:the_electric_double_layer}

\par When electric fields are applied to ionic solutions, the electrodes will attract ions of opposite charge. With equilibrium to thermal forces, the applied field will result in the electric double layer (figure \ref{fig:electric_double_layer}) \cite{ishai_electrode_2013}. The electric double layer creates an effective capacitance at the electrode-fluid interface. For electrodes with large surface areas, the capacitance may be large enough to be negligible, but for MEMS with micro-scale electrodes, the capacitance will be small and can mask the impedance of the device under testing for frequencies up to 100 kHz to 1 MHz \cite{bordi_reduction_2001}. 

\subsection{Theoretical Capacitance}

%%%
% Add in : Although several theoretical formulas attempt to approximate the the electric double layer, there are no physical descriptions that adequately account for the electric double layer. 
%%%

\begin{figure}[h]
    \centering
    \includegraphics[width=0.7\textwidth]{images/electricDoubleLayer.png}
    \caption[Stern model of the electric double layer]{The Stern model of the electric double layer that includes the Helmholtz and diffuse layers.}
    \label{fig:electric_double_layer}
\end{figure}

\par Helmholtz was the first to describe the electric double layer and modeled it as a single layer of ions adsorbed to the surface of the electrode \cite{ishai_electrode_2013}. If the ions are treated as point charges, the Helmholtz model can be interpreted as a parallel capacitor with a distance $d_H$ between the plates that represents the distance between the center of the ion and the surface of the electrode. The capacitance per unit area can be approximated as

\begin{equation}
    C = \frac{\epsilon_o\epsilon_r}{d_H},
\end{equation}

\noindent where $\epsilon_o$ is the permittivity of the vacuum, $\epsilon_r$ is the relative permittivity of the medium, and $d_H$ is the thickness of the Helmholtz layer \cite{_gongadze.pdf_????}. The Helmholtz model neglects the thermal, concentration, and voltage dependencies that experiments have confirmed \cite{_jes2011.pdf_????}.

\par Guoy and Chapman expanded on the electric double layer by including the effects of thermal motion, concentration, and applied voltage \cite{chapman_li._1913,_gongadze.pdf_????}. This results in a diffuse layer that consists of counter-ions and co-ions. The distribution of ions in the diffuse layer can be described by the Boltzmann distribution:
\begin{equation}
    n_{i\pm} = n_i^o \text{exp}\Big(\frac{ \mp z_i e\phi}{k_b T}\Big),
\end{equation}

\noindent where $n_i$ is the concentration of the ion $i$, $\phi$ is the electric potential, $T$ is the temperature, $k_b$ is the Boltzmann constant, $z_i$ is the charge number of the ion, and $e$ is the charge of an electron. The charge density can be written as the sum of all ions:
\begin{equation}
    \rho(x) = e \sum_i n_i^o z_i \text{exp}\Big(\frac{-z_i e\phi}{k_b T}\Big).
\end{equation}

\par Combining Poisson's equation with the charge distribution above, results in the Poisson-Boltzmann equation:
\begin{equation}
    \nabla^2\phi = \frac{e}{\epsilon_o\epsilon_r} \sum_i n_i^o z_i \text{exp}\Big(\frac{-z_i e \phi}{k_b T}\Big),
    \label{eqn:pb_equation}
\end{equation}

\noindent and then recognizing that
\begin{equation}
    \frac{d^2\phi}{dx^2} = \frac{1}{2}\frac{d}{dx}\Big(\frac{d\phi}{dx}\Big)^2,
\end{equation}

\noindent then with the boundary condition for electrodes far apart, $\phi\to0$ and $\frac{d\phi}{dx}\to 0$ as $x\to\infty$, equation \ref{eqn:pb_equation} can be integrated to
\begin{equation}
    \Big(\frac{d\phi}{dx}\Big)^2 = \frac{2k_b T}{\epsilon_o\epsilon_r} \sum n_i^o \bigg[exp\Big(\frac{-z_i e\phi}{k_b T}\Big) - 1\bigg].
\end{equation}

\noindent For a symmetrical electrolyte, the Poisson-Boltzmann equation can be expressed as
\begin{equation}
    \frac{d\phi}{dx} = \frac{8k_bTn^o}{\epsilon_o\epsilon_r}\sinh\Big(\frac{ze\phi}{2k_bT}\Big).
    \label{eqn:edl_efield}
\end{equation}

\noindent Capacitance can generally be defined as the differential capacitance:
\begin{equation}
    C_{diff} = \frac{d\sigma}{d\phi_o}
\end{equation}

\noindent where $\sigma$ is the surface charge of the electrode and $\phi_o$ is the electric surface potential. From Gauss's law and equation \ref{eqn:edl_efield}, the charge for the electric double layer is
\begin{equation}
    \sigma = \epsilon_o\epsilon_r \Big[\frac{d\phi}{dx}\Big]_{x=0} = \sqrt{8k_bTn_i^o\epsilon_o\epsilon_r}\sinh\Big(\frac{ze\phi_o}{2k_bT}\Big).
    \label{eqn:edl_surface_charge}
\end{equation}

\noindent Differentiating \ref{eqn:edl_surface_charge} with respect to surface potential gives the differential capacitance of the electric double layer:
\begin{equation}
    C_{GC} = \Big(\frac{2z^2e^2n_i^o\epsilon_r\epsilon_o}{k_bT}\Big)\cosh\Big(\frac{ze\phi_o}{2k_bT}\Big).
\end{equation}

\par Stern combined the Helmholtz and the Guoy-Chapman thoeries to create a model with two layers: the inner layer known as the Stern layer, which consists of the adsorbed ions on the electrode surface from the Helmholtz model, and the outer diffuse layer (figure \ref{fig:electric_double_layer} \cite{_jes2011.pdf_????}. The capacitance of the Stern model can be approximated by the capacitance of the Helmholtz and Guoy-Chapman models in series \cite{_gongadze.pdf_????}:

\begin{equation}
    C_s = \bigg[ \frac{1}{C_H} + \frac{1}{C_{GC}}\bigg]^{-1}.
\end{equation}

\par Unfortunately, these models do not fully describe the electric double layer and significantly overestimates the electric double layer capacitance. Gongadze et al. measured the capacitance of the electric double layer in a phosphate-buffered electrolyte solution with titanium electrodes. Using theory, the electric double layer capacitance was calculated as 231.7 $\mu$F/cm$^2$, 77.16 $\mu$F/cm$^2$, and 57.92 $\mu$F/cm$^2$ for the Helmholtz, Guoy-Chapman, and Stern model respectively, but the experimental capacitance was 6 $\mu$F/cm$^2$ \cite{_gongadze.pdf_????}. Predicting the electric double layer is convoluted by the surface properties of the electrodes, electrochemical reactions, and other non-quantified phenomena in the solution. 

\subsection{Correction for the Electric Double Layer}

\par To overcome the theoretical shortcomings, devices can be designed to minimize the effect of the electric double layer and/or an empirical corrective function can be applied to data. Physical compensations include four electrode sample cells, increasing the surface area by coating the electrode in black platinum or polypyrrole polystyrenesulphonate, and high current density methods \cite{ishai_electrode_2013}. However, due to a combination of device constraints, difficulty in implementation, reduced electrode strength, and failure to completely bypass the electric double layer, physical compensations are usually avoided \cite{ishai_assessment_2012}. 

\begin{figure}
    \centering
    \includegraphics[width=0.5\textwidth]{images/edl_cap_equiv.png}
    \caption{A series resistor and capacitor equivalent circuit of the electric double layer.}
    \label{fig:edl_cap_equiv}
\end{figure}

\par Empirical recalculations involve fitting a reference measurement to a function or equivalent circuit. One of the more common equivalent circuit models of the electric double layer is a resistor and capacitor in series (figure \ref{fig:edl_cap_equiv}) \cite{feldman_fractal-polarization_1998}. The impedance of the system can be expressed as
\begin{equation}
    Z = 2Z_p + Z_m,
\end{equation}
\begin{equation}
    Z_p = \frac{1 + sC_pR_p}{sC_p},
\end{equation}
\begin{equation}
    Z_m = \frac{R_m}{sC_mR_m+1},
\end{equation}

\noindent where $Z_p$, $C_p$, and $R_p$ are the impedance, capacitance, and resistance of the electric double layer respectively; and $Z_m$, $C_m$, and $R_m$ are the impedance, capacitance, and resistance of the medium respectively. $s$ is the Laplace variable, and since our interest is in the steady state behaviour, $s = jw$.

\begin{figure}[h]
    \centering
    \includegraphics[width=0.5\textwidth]{images/edl_recap_equiv.png}
    \caption{A recap equivalent circuit of the electric double layer.}
    \label{fig:edl_recap_equiv}
\end{figure}


\par A second equivalent circuit involves a similar system, but with the inclusion of recap elements (figure \ref{fig:edl_recap_equiv}) \cite{feldman_fractal-polarization_1998-1}. A recap element can be modeled as 
\begin{equation}
    C_v = R(RC)^{-v}s^{-v},
\end{equation}

\noindent where C is capacitance, R is resistance, and $v$ is $0<v<1$. $v$ determines the degree of capacitance or resistance behavior, with $v=1$ a complete capacitor, and $v=0$ a pure resistor. The recap element is known as a type of constant phase element (CPE), which is named after its property of contributing a constant angle and describes an imperfect dielectric. Constant phase elements are found in a wide range of electrodes and are used extensively in empirical descriptions of electrode polarization \cite{ishai_electrode_2013}. To fit data to the CPE, it is useful to express the system impedance as
\begin{equation}
    Z_{sys} = Z_{bulk} + Z_0\bigg(j\frac{f}{f_0}\bigg)^{-v},
\end{equation}

\noindent where $Z_{bulk}$ is the impedance of the medium, $j$ is the imaginary number, $f$ is frequency, $f_0$ is the onset frequency of the electrode polarization, and $Z_0$ is a impedance fitting term. It is convenient to find the onset frequency by looking for the characteristic tail of the CPE on a nyquist plot (figure \ref{fig:nyquist_onset}).

\begin{figure}[h]
    \centering
    \includegraphics[width=\textwidth]{images/nyquist_ep_onset.png}
    \caption[Electrode polarization on a Nyquist plot]{Electrode polarization on a Nyquist plot with Z' and Z'' the real and imaginary components of the impedance respectively. The onset frequency of the electrode polarization can be identified by the beginning of the characteristic CPE tail \cite{ishai_assessment_2012}.}
    \label{fig:nyquist_onset}
\end{figure}


%%ist %%%%%%%%%%%%%%%%%%%%%%%%%%%%%%%%%%%%%%%%%%%%%%%%%%%%%%%%%%
 % Previous Work on the Cal Poly Biofluidic Lab's EIS System
 %%%%%%%%%%%%%%%%%%%%%%%%%%%%%%%%%%%%%%%%%%%%%%%%%%%%%%%%%%%%%%%%%%%
 %\section{Previous Work on the Cal Poly Biofluidic Lab's EIS System}
 
 % In 2009 Josh Fadriuela and Stephanie Hernandez fulfilled their thesis under Dr.Clague to create a cell impedance sensor system. Their work will be the foundation for this thesis \cite{fadriquela_design_2009-1}, \cite{hernandez_single_2009-1}.
 