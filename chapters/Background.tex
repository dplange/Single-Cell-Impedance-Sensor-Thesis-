

%%%%%%%%%%%%%%%%%%%%%%%%%%%%%%%%%%
% Cells
%%%%%%%%%%%%%%%%%%%%%%%%%%%%%%%%%%
\section{Cells}
\par The cell is the basic unit of life. At the fundamental level, a cell must have an outer membrane that acts as a gatekeeper between the surrounding environment and the interior constituents. The actual contents of the cell interior vary widely between between the classes of prokaryotes (single-celled organisms) and eukaryotes (cells of multi-celled organisms), and even between different instances of the classes and different states of those instances. For example, mature red blood cells do not contain a nuclei or any genetic material, but skeletal muscle fiber consist multiple nuclei. However, all these variations can be generalized to useful cell models, such as the model for human cells in figure \ref{fig:human_cell_model}.   
\begin{figure}[ht]
 \centering
 \includegraphics[width=\textwidth]{images/humanCellOverview.png}
 \caption[Diagram of human cell structure.]{Diagram of human cell structure. \cite{daniel_d_chiras_human_2005} }
 \label{fig:human_cell_model}
 \end{figure}
 
 
 \par Generally, a cell carries genetic information in the nucleus inside the cell membrance and cytoplasm. Cytoplasm is all cell material inside of the cell membrane except the nuclei. The cytoplasm consists of non-nuclei organelles and cytosol.  Cytosol refers to the cellular solution between the cell organelles and the membrane, wich contains salts, nucleic acids and cytoskeleton filaments. Organnelles are membrane bound structures inside the cell that perform a special function. Organelles include nuclei, mitochondria, the Golgi apparatus, lysosomes, and vacuoles.  
 
 \par All of these 
 
 %%%%%%%%%%%%%%%%%%%%%%%%%%%%%
 % The Cell Membrane
 %%%%%%%%%%%%%%%%%%%%%%%%%%%%%
 \subsection{The Cell Membrane}
 \par The cell membrane is an essential component of the cell that is necessary for a number of critical functions. The cell membrane is primarily composed of the lipid bilayer which is a semi-permeable barrier constructed from phospholipids. A phospholipid has a hydrophilic head group and a hydrophobic tail group. When exposed to an aqueous environment, phospholipids join together into a configuration that leaves the head groups exposed to the aqueous solution and the tail group surrounded by other non-polar tail groups. In the case of a cell, the phospholipids take the form of a spherical bilayer membrane (figure \ref{fig:cell_membrane}). Although these lipids prevent the passage of many molecule, the form an imperfect semi-permeable membrane that allows for some molecules o pass into or out of the cell. 
 
 \par In addition to the phospholipids the membrane contains many other constituents that play critical roles. Cell membranes are often packed with proteins that are critical for may functions. Proteins that stick out of both sides of the membrane are known as transmembrane proteins. These proteins range from channel proteins tat allow passage of elements that could not normally make through the membrane, to proteins involved in signalling. 
 
 \par The membrane is the final enforcer of the equilibrium of the cell. 
 \begin{figure}[h]
    \centering
    \includegraphics[width=\textwidth]{images/Cell_membrane_detailed_diagram.png}
    \caption[Diagram of the cell membrane]{Diagram of the cell membrane \cite{mariana_ruiz_cell_????}}
    \label{fig:cell_membrane}
 \end{figure}
 
 
 %%%%%%%%%%%%%%%%%%%%%%%%%%%%%%%%%%%%%%%%%
 % Electrical Model of the Cell
 %%%%%%%%%%%%%%%%%%%%%%%%%%%%%%%%%%%%%%%%%
 \subsection{Electrical Model of the Cell}
 
 
 %%%%%%%%%%%%%%%%%%%%%%%%%%%%%%%%
 % Dielectric Spectroscopy
 %%%%%%%%%%%%%%%%%%%%%%%%%%%%%%%%
 \section{Dielectric Spectroscopy}

 % Read into articles and provide more details
 
 \par The dielectric properties of cells have been investigated since 1910 when H\"{o}ber showed the existence of the cell membrane by measuring the conductivity of erythrocytes (red blood cells) at high and low frequencies \cite{hober_r_methode_1910}. The field of study further developed with Fricke's application of Maxwell's equations to measure the capacitance and thickness of the cell membrane in 1924-1925 \cite{james_clerk_maxwell_treatise_1892, fricke_h_mathematical_1924, fricke_h_electric_1924, fricke_h_electric_1931}. 
 
 \par Then in 1998, Cole used Maxwell's mixture equation to derive the complex impedance of a single shelled cell model and developed equations to describe the Cole-Cole plot \cite{cole_electric_1928}. And with Curtis, Cole made the first single cell measurements on a Nitella cell \cite{curtis_transverse_1937}. A Nitella cell is a large bacteria cell that ranges in length from 20 $\mu$m to 60 mm. 
 
 \par From 1957-1968 Schwan used broadband electric impedance spectroscopy to identify $\alpha$, $\beta$, and $\gamma$ dispersions of a cell \cite{schwan_h_p_electrical_1957,schwan_h_p_electrical_1963,schwan_electrical_1994}. Where a dielectric dispersion is a frequency dependent relationship between an applied electric field and the permittivity of a material. The permittivity is the resistance to forming an electric field over a medium, and can be defined as
 \begin{equation}
     \textbf{D} = \boldsymbol{\epsilon} \textbf{E} = \epsilon_0 \textbf{E} + \textbf{P}
     \label{eqn:dielectric_displacement}
 \end{equation}
 
 % Include a discussion of the polarization tensor?
 \noindent where $\boldsymbol{\epsilon}$ is the second order permittivity tensor, $\epsilon_0$ is the permittivity of the vacuum, $\textbf{E}$ is the first order electric field tensor, $\textbf{P}$ is first order polarization density tensor which represent the density of permanent or induced electric dipole moments, and $\textbf{D}$ is the first order electric displacement tensor and represents how the electric field affects the organization or movement of charges in a medium. For a linear, homogeneous, isotropic material with instantaneous charge response, equation \ref{eqn:dielectric_displacement} can be simplified to
 \begin{equation}
    \textbf{D} = \epsilon \textbf{E}
 \end{equation}
 
 \noindent where $\epsilon$ is the zeroth permittivity tensor. 
 
 \par Examples of charge reorganization include electron cloud distortion, charged particle displacement, and reorientation of molecules with dipoles. $\alpha$, $\beta$, and $\gamma$ dispersions refers to the ranges of alternating current frequencies where the permittivity decreases significantly in cells (i.e. dielectric relaxation). Figure \ref{fig:schwan_dispersions} depicts an approximate permittivity spectrum of a cell.
 
 \begin{figure}[ht]
 \centering
 \includegraphics[width=0.7\textwidth]{images/schwanDispersions.png}
 \caption[General cell permittivity spectrum]{General cell permittivity spectrum with labeled $\alpha$, $\beta$ and $\gamma$ dispersions.}
 \label{fig:schwan_dispersions}
 \end{figure}
 
 \par $\beta$ dispersion occurs up to 10 Mhz and is caused by the relaxation of the cell membrane. For lower frequencies, the membrane has enough time to charge and provide resistance to the electric field, but at frequencies greater than about 10 Mhz, the membrane does not have sufficient time to fully charge and provides little resistance to the electric field. $\gamma$ dispersion is related to the relaxation of water molecules. Instead of a charging membrane, $\gamma$ dispersion is due to the dipole rotation of the of water molecules and the biomolecules that water is bound to. The source of $\alpha$ dispersion is undetermined, but current theories include the surface reactance of the electric double layer near the charged cell, the impedance of the sarcoplasmic recticulum (for muscle fibers), or the frequency dependent conductance of ionic membrane channels as predicted by the Hodgkin-Huxley equations \cite{schwan_electrical_1994}.
 
 
 \par These developments laid the foundations for the following techniques for measuring the dielectric properties of single cells. 
 
 %%%%%%%%%%%%%%%%%%%%%%%%%%%%%%%%
 % Dielectrophoresis
 %%%%%%%%%%%%%%%%%%%%%%%%%%%%%%%%
 \subsection{Dielectrophoresis}
 \par The dielectrophoretic force (DEP) is the force generated on a particle suspended in a solution by the interaction of an applied nonuniform electric field and an induced dipole moment developed through Maxwell-Wagner polarization (a build up of charges at dielectric boundaries). Peter Debye and Herbert A. Pohl were among the first to develop dielectrophoresis and demonstrated how particles could be moved with nonuniform electric fields \cite{muller_potential_2003}. With an applied AC electric field, the average DEP force on a spherical particle is expressed as \cite{morgan_single_2007, green_dielectrophoresis_1999}
 
 \begin{equation}
     \big< F_{DEP} \big> = \pi \epsilon_m R^3 \text{Re}[\tilde{f}_{CM}] \nabla |\textbf{E}|^2 
     \label{eqn:dep_force}
 \end{equation}
 
 \noindent with
 
 \begin{equation}
     \tilde{f}_{CM} = \frac{\tilde{\epsilon}_p - \tilde{\epsilon}_m}{\tilde{\epsilon}_p + 2\tilde{\epsilon}_m} 
     \label{eqn:fcm_background}
 \end{equation}
 
 \noindent where $\epsilon_m$ is the permittivity of the medium, $R$ is the radius of the particle, $\tilde{f}_{CM}$ is the Clausius-Mossotti factor for spherical particles,  $\tilde{\epsilon}_p$ is the complex permittivity of the particle, $\tilde{\epsilon}_m$ is the complex permittivity of the medium, and $\textbf{E}$ is the electric field. For most cases, the complex permittivity can be expressed as 
 \begin{equation}
     \tilde{\epsilon} = \epsilon - j\frac{\sigma}{\omega}
 \end{equation}
\noindent where $\epsilon$ is the permittivity, $j = \sqrt{-1}$, $\sigma$ is the conductivity, and $\omega$ is the angular frequency. 

\par From equation \ref{eqn:dep_force} and \ref{eqn:fcm_background}, it can be noted that, given an electric field, the magnitude of the dielectrophoretic force depends on the size of the particle and the polarizability of the particle with respect to the medium (i.e Re$\big[\tilde{f}_{CM}\big]$), and the sign of the force only relates to the polarizability of the particle and medium (figure \ref{fig:freq_crossover}). It should be noted that, based on equation \ref{eqn:fcm_background}, the real part of Clausius-Mossotti factor is restricted to less than 1 and greater than -0.5. However, for non-spherical shapes, such as elongated ellipses, equation \ref{eqn:fcm_background} does not satisfy the Clausius-Mossotti factor which can reach values far greater than 1.  


\begin{figure}[ht]
 \centering
 \includegraphics[width=\textwidth]{images/DEPCrossover.png}
 \caption[Frequency crossover of Clausius-Mossotti factor] {Frequency crossover of Clausius-Mossotti factor with labeled regions of positive and negative dielectriphoretic forces. From equation \ref{eqn:dep_force}, if $\tilde{\epsilon}_p$ is greater than $\tilde{\epsilon}_m$, then the DEP force is positive, if $\tilde{\epsilon}_m$ is less than $\tilde{\epsilon}_p$, then the DEP force is negative.}
 \label{fig:freq_crossover}
\end{figure}
 
 \par To extract dielectric properties of cells using dielectric forces, the DEP crossover frequency can be experimentally determined and compared to theoretical values \cite{morgan_single_2007}. The crossover frequency occurs at the frequency when the real part of the Clausius-Mossotti factor equals zero and is where the DEP force switches from a positive to a negative DEP force (or \textit{vice versa}). Experimentally, this can be determined by the observation of a stationary cell in a non-uniform electric field. Theoretically, the crossover frequency can be described as
 
 \begin{equation}
     f_{cross} = \frac{1}{2 \pi}\sqrt{\frac{(\sigma_m-\sigma_p)(\sigma_p+2\sigma_m)}{(\epsilon_p-\epsilon_m)(\epsilon_p+2\epsilon_m)}},
     \label{eqn:f_cross}
 \end{equation}
  
 \noindent where $f_{cross}$ is the crossover frequency. Equation \ref{eqn:f_cross} can be rewritten to include the Maxwell-Wagner relaxation frequency as \cite{morgan_single_2007}
 
 \begin{equation}
    f_{cross} = \frac{1}{2 \pi} \sqrt{\frac{\sigma_m - \sigma_p}{\epsilon_p - \epsilon_m}f_{MW}}
 \end{equation}
 
 \noindent with
\begin{equation}
    f_{MW} = \frac{1}{2\pi\tau_{MW}} 
\end{equation}
\begin{equation}
    \tau_{MW} = \frac{\epsilon_p + 2\epsilon_m}{\sigma_p + 2\sigma_m}
\end{equation}
 
 \noindent where $f_{MW}$ and $\tau_{MW}$ are the Maxwell-Wagner relaxation frequency and time constant respectively. The Maxwell-Wagner relaxation frequency characterizes the frequency of $\beta$ dispersion described by Schwan \cite{morgan_single_2007}.
 
 \par For a single shelled model of a cell, the crossover frequency can be expressed as
 
 \begin{equation}
     f_{cross} = \frac{\sqrt{2}}{8\pi R C_{mem}}\sqrt{(4\sigma_m - RG_{mem})^2 - 9R^2G^2_{mem}},
 \end{equation}
 
 \noindent where $C_{mem}$ is the specific capacitance of the cell membrane and $G_{mem}$ is the specific conductance of the membrane.
 
 %%%%%%%%%%%%%%%%%%%%%%%%%%%%
 % Electrorotation
 %%%%%%%%%%%%%%%%%%%%%%%%%%%%
 \subsection{Electrorotation}
 
 \par Electrorotation is the applied torque to a polarized particle in a medium of different dielectric properties by a rotating electric field. Such a field can be created with a system of four electrodes with sinusoidal signals of 90\textdegree \;phase shifts between adjacent electrodes \cite{goater_electrorotation_1999}. The phenomenon of electrorotation was first documented by by Pohl in 1978 and then developed into dependable methods by Arnold and Zimmerman in 1982 \cite{pohl_dielectrophoresis_1978-1, arnold_rotating-field-induced_1982}.
 
 \par The average torque exerted on a spherical particle can be expressed as \cite{morgan_single_2007}
 \begin{equation}
    \Tau_{\text{ROT}} = -4\pi \epsilon_m R^3 \text{Im}\big[\tilde{f}_{CM}\big]|\textbf{E}|^2
    \label{eqn:ROT_torque}
 \end{equation}
 
 \noindent where $\Tau_{\text{ROT}}$ is the torque applied to the induced dipole of the particle, $\epsilon_m$ is the permittivity of the medium, $R$ is the radius of the spherical particle, $\textbf{E}$ is the electric field, and $\text{Im}\big[\tilde{f}_{CM}\big]$ is the imaginary part of the Clausius-Mossotti factor expressed in equation \ref{eqn:fcm_background}. Similarly to the average dielectrophoretic force (equation \ref{eqn:dep_force}), the Clausius-Mossotti factor is the frequency dependent element and controls whether the particle rotates with or against the applied rotating electric field.
 
 \par The applied torque can be measured by the angular velocity of the particle which can be expressed as \cite{morgan_ac_2003}
 \begin{equation}
     R_{\text{ROT}} = - \frac{\epsilon_m\text{Im}\big[\tilde{f}_{CM}\big]|\textbf{E}|^2}{2\eta}K, 
 \end{equation}
 \noindent where $R_{\text{ROT}}$ is the angular velocity, $\eta$ is the dynamic viscosity, and K is the scaling factor. When the angular velocity has reached equilibrium, the applied torque will be proportional $R_{\text{ROT}}$ and related by the scaling factor K. The electrorotation torque spectrum of a cell can be used, in tandem with equation \ref{eqn:ROT_torque}, to determine the dielectric properties of the cell. 
 
 \par By combining dielectrophoretic and electrorotaion spectroscopy techniques, significant details on the dielectric properties can be obtained. Since dielectrophoretic spectroscopy determines the real Clausius-Mossotti factor spectrum, and electrorotation determines the imaginary Clausius-Mossotti factor, the full $\tilde{f}_{CM}$ spectrum can deduced from which dielectric properties can be extracted. Separating the real and imaginary components of the Clausius-Mossotti factor, the following expressions are obtained \cite{hober_zweites_1912, morgan_single_2007}.
 \begin{equation}
     \text{Re}\big[\tilde{f}_{\text{CM}}\big] = \frac{(\sigma_p - \sigma_m)}{(1+\omega^2\tau_{\text{MW}}^2)(\sigma_p + 2\sigma_m)} + \frac{\omega^2\tau^2_{\text{MW}}(\epsilon_p-\epsilon_m)}{(1+\omega^2\tau_{\text{MW}}^2)(\epsilon_p+2\epsilon_m)}
 \end{equation}
 \begin{equation}
     \text{Im}\big[\tilde{f}_{\text{CM}} \big] = \frac{3\omega\tau_{\text{MW}}(\epsilon_p\sigma_m-\epsilon_m\sigma_p)}{(1+\omega^2\tau^2_{\text{MW}})(\epsilon_p+2\epsilon_m)(\sigma_p+2\sigma_m)}
 \end{equation}
 \par A major shortcoming of the previously discussed techniques is the long time needed for measurements. An electrorotaion assay could potentially take up to several seconds per test and limit the number of cells tested. An alternative approach, electric impedance spectroscopy, allows for rapid dielectric spectroscopy, and with flow-through designs, allows dielectric spectroscopy of every cell in a sample. 
 
 %%%%%%%%%%%%%%%%%%%%%%%%%%%%%%%%%%%%%%%%%%%%%
 % Electrical Impedance Spectroscopy
 %%%%%%%%%%%%%%%%%%%%%%%%%%%%%%%%%%%%%%%%%%%%%
 \subsection{Impedance Spectroscopy}
 
 \subsection*{Impedance}
 \par Electrical Impedance is defined as the total opposition of a system to the flow of an alternating current (AC). If a sinusoidal voltage is applied to an arbitrary system, then the voltage and current can be expressed as
 \begin{equation}
    V(w) = |V|\cos(wt-\theta_V)
    \label{eqn:V(w)}
 \end{equation}
 \begin{equation}
    I(w) = |I|\cos(wt - \theta_I)
    \label{eqn:I(w)}
 \end{equation}
 \noindent where $w$ is the angular frequency, $t$ is the time in seconds, and $\theta_V$ and $\theta_I$ are the phase of $V$ and $I$ respectively. Using Euler's formula, equations \ref{eqn:V(w)} and \ref{eqn:I(w)} can be recognized as the real part of the following:
 \begin{equation}
    \hat{V}(w) = |V|e^{j(wt-\theta_V)}
 \end{equation}
 \begin{equation}
     \hat{I}(w) = |I|e^{j(wt-\theta_I)}
 \end{equation}
 
 \par Applying Ohm's law, the impedance of the system can be expressed as
 
 \begin{equation}
    Z = \frac{|V|}{|I|}e^{j(\theta_I-\theta_V)}
 \end{equation}
 \noindent or as 
 \begin{equation}
     Z = |Z|e^{j\theta}
     \label{eqn:impedance_euler}
 \end{equation}
 \noindent where $\theta$ is the difference in phase of the voltage from the current after the current passes through the system. Equation \ref{eqn:impedance_euler} sets the interpretation of impedance to be a vector that lives on the complex plane where $|Z|$ is the length of the vector and $\theta$ is the angle of the vector from the positive real axis (figure \ref{fig:Impedance_diagram}).
 
 \begin{figure}[ht]
 \centering
 \includegraphics[width=0.7\textwidth]{images/impdedanceDiagram.png}
 \caption[Representation of Impedance as a vector on the complex plane]{Representation of Impedance as a vector on the complex plane with a real component $R$ and imaginary component $X$. }
 \label{fig:Impedance_diagram}
 \end{figure}
 
 \par As a vector on the complex plane, impedance can be represented in rectangular coordinates and polar form:
 \begin{equation}
    Z = R + jX,
 \end{equation}
 \noindent where $R$ and $X$ are the real and imaginary components of the impedance vector, 
 \begin{equation}
     Z = |Z| \angle \theta
     \label{eqn:z_polar}
 \end{equation}
 \noindent where $|Z|$ is the magnitude of the impedance vector and $\theta$ is the angle of the vector from the positive real axis. Given an AC applied signal, Ohm's law can then be expressed as
 \begin{equation}
    |V| = |Z||I|
    \label{eqn:ohm_mag}
 \end{equation}
 \noindent and the phase of the voltage signal will be shifted from the current signal by $\theta$. 
 
 \begin{figure}[ht]
    \centering
    \includegraphics[width=0.9\textwidth]{images/ac_signal.png}
    \caption[Current Response of a system to an applied voltage]{Current Response of a system to a 1 volt applied potential. Using equations \ref{eqn:z_polar} and \ref{eqn:ohm_mag}, the impedance of the system can be expressed as $Z=\sqrt{2}\angle -\dfrac{\pi}{4}$} 
    \label{fig:ac_signal}
 \end{figure}
 
 \par The rectangular form and polar form of the impedance are related as follows:
 \begin{equation}
     |Z| = \sqrt{R^2 + X^2}
 \end{equation}
 \begin{equation}
    \theta = \text{arctan}(\frac{X}{R})
 \end{equation}
 \begin{equation}
    R = |Z|\cos\theta
 \end{equation}
 \begin{equation}
     X = |Z|\sin\theta
 \end{equation}
 
 \par The impedance of a system can be represented by a combination of elementary circuit components: reistors, capacitors, and inductor. The voltage response of these elements in the time domain can be expressed as
 \begin{alignat}{2}
    &\text{Resistor:} \quad  &&V(t) = I(t)R \label{eqn:ohms_law}\\\nonumber\\
    &\text{Capacitor:} \quad &&V(t) = C \frac{dV}{dt}\\\nonumber\\
    &\text{Inductor:} \quad  &&V(t) = L \frac{dI}{dt}\\\nonumber
 \end{alignat}
 \noindent where $V(t)$ and $I(t)$ are the voltage and current as a function of time, $R$ is the resistance, $C$ is the capacitance, $L$ is the inductance. After applying Laplace transforms, the frequency dependent impedance of each element can be obtained as
  \begin{alignat}{2}
    &\text{Resistor:} \quad  &&Z(w) = R \;\;\;\\\nonumber\\
    &\text{Capacitor:} \quad &&Z(w) = \frac{1}{jwC}\\\nonumber\\
    &\text{Inductor:} \quad  &&Z(w) = jwL\\\nonumber
 \end{alignat}

\subsection*{Spectroscopy: Coulter Counter}
\par One of the first devices capable of measuring the electrical properties of particles is the Coulter counter, which measures the direct current resistance of two fluid filled chamber connected by an aperture (figure \ref{fig:coulter_counter}). The apparatus drives fluid through the aperture by a pressure differential caused by different levels of fluid in the two chambers. As particles flow through the aperture, conductive fluid is displaced and there is a change in resistance. Measuring the resistance over time, pulses of increased resistance marks a particle flowing through the aperture, and the magnitude of the resistance pulse will correspond to the size of the particle. 


\begin{figure}[ht]
    \centering
    \includegraphics[width=0.8\textwidth]{images/coultierCounter.png}
    \caption[Illustration of Coulter counter principles]{A Coulter counter that measures the DC resistance of the fluid in the chamber with two electrodes. When a particle enters the aperture, the change in resistance will provide information about the size and electrical properties of the particle}
    \label{fig:coulter_counter}
\end{figure}


In 1970, Deblois and Bean developed an analytical model of the Coulter counter resistance that can be related to dielectric properties of the particle \cite{deblois_counting_1970}. The resistance of the large chambers were assumed to be negligible and the resistance of the aperture was derived (figure \ref{fig:aperture}). For a tube of a mixture solution, the resistance can be expressed as

\begin{equation}
    R_t = \frac{4\rho_{mix}L}{\pi d_{t}} G,
    \label{eqn:tube_resistance}
\end{equation}

\noindent where $G$ is a correction term for the non-uniform current density, $R_t$ is the resistance of the tube, $L$ is the length of the tube, $d_{t}$ is the diameter of the tube, and $\rho_{mix}$ is the resistivity of the mixture, which can be expressed using Maxwell's approximation, under the assumption that the particle diameter is small relative to the tube, as

\begin{equation}
    \rho_{mix} = \rho_m\bigg[ 1 + \frac{3\phi}{2} + \Big(\frac{-9\phi \rho_m}{4 \rho_p + 2 \rho_m}\Big)\bigg]
\end{equation}
\noindent where $\phi$ is the volume fraction, $\rho_m$ is the resistivity of the conductivity, and $\rho_p$ is the resistivity of the particle. The volume fraction of a spherical particle in a tube is

\begin{equation}
    \phi = \frac{2d_p^3}{3d^2_tL}
\end{equation}

\noindent where $d_p$ is the diameter of the particle.

\begin{figure}[ht]
    \centering
    \includegraphics[width=0.7\textwidth]{images/aperture.png}
    \caption[The aperture connecting the two chamber of the Coulter counter.]{The aperture connecting the two chamber of the Coulter counter where $L$ is the length of the aperture, $d_p$ is the diameter of the particle flowing through the aperture, and $d_{t}$ is the diameter of the aperture.}
    \label{fig:aperture}
\end{figure}
 
\par Since 2000, Coulter counters have been fabricated on the micro scale and has allowed increased sensitivity and adaptability, leading to nanopore devices capable of analyzing single molecules \cite{sun_single-cell_2010}.
 
 \subsection*{Spectroscopy: Microfluidic Impedance Cytometry}
 \par Instead of electrodes on either side of an aperture, as in Coulter counter designs, microfluidic impedance cytometry designs electrodes directly onto the walls of the test volume. The test volume is the region that contains the particle and fluid that contain significant current density. The impedance of the test volume can be approximated using Maxwell's mixture theory and a geometric factor, and can be expressed as 
 
 \begin{equation}
       \tilde{Z}_{mix} = \frac{1}{jw\tilde{C}_{mix}}
    \label{eqn:impedance_with_cap}
 \end{equation}
 
 \noindent where $\tilde{C}_{mix}$ is the complex capacitance of the test volume, and can be expressed as
 
 \begin{equation}
     \tilde{C}_{mix} = \tilde{\epsilon}_{mix}G_f
 \end{equation}
 
 \noindent where $\tilde{\epsilon}_{mix}$ is the complex permittivity derived from Maxwell's mixture law and $G_f$ is the geometric factor that accounts for fringing electric fields. Further details are provided in section \ref{sec: analytical_impedance_solution}.
 
 \begin{figure}[ht]
     \centering
     \includegraphics[width=\textwidth]{images/parallel.png}
     \caption{Parallel electrode configuration for impedance cytometry}
     \label{fig:parallel_electrodes}
 \end{figure}
 
 \par Two of the more common electrode configurations are parallel facing electrodes and coplanar electrodes (figure \ref{fig:parallel_electrodes} and \ref{fig:coplanar_electrodes}). Coplanar electrodes are easier to fabricate than parallel facing electrodes, but come at the cost of diminished sensitivity and accuracy \cite{sun_analytical_2007}. Parallel electrode systems experience smaller variations in the produced electric field in the direction orthogonal to the electrode plane when compared to coplanar electrodes. As a result the signal variation based on the height of the cell in the test volume will be greater for coplanar configurations. The will result in a loss of accuracy, unless the vertical placement of the cell in the test volume can be controlled. Parrallel facing electrodes allow designs that can create a large volume fraction for a test volume with a single particle, and therefore, create a system with enhanced sensitivity. This can be observed by noting that given an electrode width, channel height, and channel depth, the coplanar test volume will be over twice that of parallel electrodes configurations.
  
 
 \begin{figure}[ht]
     \centering
     \includegraphics[width=\textwidth]{images/coplanar.png}
     \caption{Coplanar electrode configuration for impedance cytometry}
     \label{fig:coplanar_electrodes}
 \end{figure}
 
 \par In microfluidic impedance cytometry, an alternating currents at a series of frequencies are applied to the system to see the impedance response over a range of frequencies. From this impedance spectrum, the frequency dependent complex response can be depicted in magnitude-phase plots, real-imamginary vs frequency plots, and nyquist plots in order to gain insight into the behaviour for the impedance response (figure \ref{fig:impedance_diagrams}). 
 

\begin{figure}[ht]
    \centering
    \includegraphics[width=0.3\textwidth]{images/exampleCircuit.png}
    \caption{Example circuit used in impedance diagrams}
    \label{fig:example_circuit}
\end{figure} 
 
 
 \begin{figure}
    \centering
    \begin{subfigure}[b]{\textwidth}
        \centering
        \includegraphics[width=0.68\textwidth]{images/magPhaseExample.png}
        \caption{Magnitude-phase versus frequency diagrams}
        \label{fig:mag_phase_ex}
    \end{subfigure}
 
    \begin{subfigure}[b]{\textwidth}
        \centering
        \includegraphics[width=0.68\textwidth]{images/complexRealExample.png}
        \caption{Real-imaginary versus frequency diagrams}
        \label{fig:real_complex_ex}
    \end{subfigure}
    \begin{subfigure}[b]{\textwidth}
        \centering
        \includegraphics[width=0.52\textwidth]{images/nyquistExample.png}
        \caption{Nyquist plot}
        \label{fig:nyquist_plot}
    \end{subfigure}
    \caption[Impedance Diagrams]{Examples of magnitude-phase diagrams, real-complex diagrams, and a nyquist plot. The data is based off of the impedance of the circuit in figure \ref{fig:example_circuit}}
    \label{fig:impedance_diagrams}
\end{figure}

\par The 
 
 \subsection*{Circuit Implementation}
 \par The application of impedance spectroscopy has a broad range from analysis of reaction kinetics, the study of batteries, to the research of biological samples. The general process of impedance spectroscopy involves the application of a voltage or current signal and the measurement of the signal response over a range of frequencies to obtain an impedance spectrum of the device under testing (DUT). The resulting spectrum provides insight into the dielectric properties of the DUT. 
 
 \par The simplest system for measuring impedance is the I-V method (figure \ref{fig:IV_impedance_measurement}). The I-V method determines the impedance of a DUT by measuring the voltage drop over the DUT and a high precision resistor in series. The current through the system can be calculated with Ohm's law at the high precision resistor. The impedance of the DUT for the I-V method can be expressed as 
 
 \begin{equation}
     Z_{DUT} = \frac{\Delta V_{DUT}}{I} = \frac{V_1 - V_2}{V_2}R
     \label{eqn:IV_Z}
 \end{equation}
 
 \noindent where $V_1$, $V_2$, and $R$ correspond to values of elements in figure \ref{fig:IV_impedance_measurement}. An important disadvantage of this configuration, is that the accuracy of the impedance values relies on the precision of the resistor value, and quality of the probes measuring $V1$ and $V2$ which can be challenged under large DUT impedances and high frequencies.
 
 \begin{table}[ht]
    \centering
    \includegraphics[width=\textwidth]{images/impedanceMeasurementMethods.png}
    \caption[Common impedance measurement methods]{Common impedance measurement methods \cite{keysight_technologies_impedance_2015}}
    \label{tab:z_measurement_methods}
 \end{table}
 
 \begin{figure}[ht]
    \centering
    \includegraphics[width=0.7\textwidth]{images/I-VMethod.png}
    \caption[I-V impedance measurement configuration]{I-V impedance measurement }
    \label{fig:IV_impedance_measurement}
\end{figure}

\par Another circuit configuration to measure impedance is the auto-balancing bridge method (figure \ref{fig:auto-balancing_bridge}) \cite{keysight_technologies_impedance_2015}. The auto-balancing bridge method works as an extension of the I-V method. Instead of measuring the current directly, the potential on the low side of the DUT is driven to a virtual ground by an op-amp, and an equivalent current is measured. The impedance of the DUT for auto-balancing bridge can be expressed as

\begin{equation}
    \frac{V_1}{Z_{DUT}} = I_1 = I_2 = \frac{V_2}{R}
\end{equation}
\begin{equation}
    Z_{DUT} = \frac{V_1}{I_1} = \frac{V_1}{V_2}R
\end{equation}

Key advantages of the auto-balancing bridge method, is that the input impedance of the I-V converter portion of the circuit essentially becomes zero and does not affect measurements, and distributed capacitance 

 \begin{figure}[ht]
    \centering
    \includegraphics[width=\textwidth]{images/autoBalancingBridge.png}
    \caption[Auto-balancing bridge impedance method]{Auto-balancing bridge impedance method}
    \label{fig:auto-balancing_bridge}
\end{figure}
 
 % Electrical impedance spectroscopy has historically only been used on multiple cells to obtain aggregate data, however, with the rise of microelectomechanical systems (MEMS) and microfluidics, electrical impedance spectroscopy can be applied to individualy cells.
 
 
 %%%%%%%%%%%%%%%%%%%%%%%%%%%%%%%%%%%%%%%%%%%%%%%%%%%%%%%%%
 % Microelectromechanical Systems and Microfluidics
 %%%%%%%%%%%%%%%%%%%%%%%%%%%%%%%%%%%%%%%%%%%%%%%%%%%%%%%%%
 \section[MEMs and Microfluidics]{MEMS and Microfluidics}
 
 \par Microelectromechanical systems (MEMS) are devices on the order of microns, smaller than the diameter of a human hair (about 100 microns). MEMS evolved from the manufacturing processes used to fabricate integrated circuits and ink jet cartridges \cite{xia_soft_1998-1}. Using silicon based microfabrication techniques such as photolithography, etching, and metal deposition, mechanical and electrically driven micro-sized pumps, cantilevers, and sensors can be fabricated \cite{wang_bio-mems:_2006}. 
 
 \par Using these microfabrication techniques, system can be miniturized in precise and reproduceable packages with applications in a variety of biology related problems. MEMS applied in these fields are refereed to as BioMEMS, and is an area of rapidly growing interest and research \cite{grayson_biomems_2004}. The majority of BioMEMS are focused on creating diagnostic systems that can identify diseases and properties of biogical substances. These devices usually need to treat, filter, and utilize detection methods that include electrophoresis, dielectrophoresis, surface plasmon resonance, and !!!! Need better examples / rewording !!!. In order to run diagnostics on a solution of biological material, a sample usually needs to undergo pretreatment, sample preparation, preconcentration and detection. With the miniaturization afforded by microfabrication technology, the concept of combining these steps into a single chip is now feasible. These devices are refereed to as labs on a chip or micro total analysis systems.
 
 %%%%%%%%%%%%%%%%%%%
 % MEMS
 %%%%%%%%%%%%%%%%%%%
 \subsection{MEMS}
 
 
 %%%%%%%%%%%%%%%%%%%%%%%%%%
 % Microfluidics
 %%%%%%%%%%%%%%%%%%%%%%%%%%
 \subsection{Microfluidics}
 
\par In most BioMEMS there is fluid flow on the micrometer scale. At this scale fluid flow physics differs from fluid flow at the macro scale. Understanding and leveraging microfluidic mechanics allows the advantage of smaller reagent and sample volumes, multiplexing, and physic phenomenom that allow experiments and functions not possible at the macro scale. Microfluidics can be defined as the study and application of fluids in structures on the micrometer scale \cite{pamb}. Laminar flow, diffustion, fluidic resistance, surface area to volume, and surface tension, may not be dominant in phenomenom on the macro scale, but on the micro scale become dominant \cite{pamb-12}. 

\subsection*{The Navier-Stokes Equation}

\par !!!!!!!! Add book citation !!!!!!!!!!!!!!

\par The Navier-Stokes formula is a partial differential equation that describes the fluid velocity given a set of boundary conditions. The Navier-Stokes equation is derived from applying the conservation of momentum to an infinitesimally small arbitrary control volume in fluid:

\begin{equation}
    \rho \frac{\text{D}\textbf{u}}{\text{D}t} = \sum \textbf{f}
\end{equation}

\noindent where $\rho$ is the mass density, $\textbf{u}$ is the velocity vector, $\sum \textbf{f}$ is the sum of forces applied to the control volume, and $\frac{\text{D}()}{\text{D}t}$ is the material derivative. The material derivative is the time derivative of a function that is spatially dependent and arises from the chain rule. The material derivative of an arbitrary spatial and temporal function ($F = F(t,x,y,z)$) can be calculated as

\begin{equation}
    \frac{\text{D}F}{\text{D}t} = \frac{dF}{dt} + \frac{dF}{dx}\frac{dx}{dt} + \frac{dF}{dy}\frac{dy}{dt} + \frac{dF}{dz}\frac{dz}{dt},
\end{equation}

and can be expressed as 

\begin{equation}
    \frac{\text{D}F}{\text{D}t} = \frac{dF}{dt} + \textbf{v} \cdot \boldsymbol{\nabla}F,
\end{equation}

or in index notation as

\begin{equation}
    \frac{\text{D}F}{\text{D}t} = \frac{dF}{dt} + v_j\frac{dF}{dx_j}.
\end{equation}

\par Surface forces are present in many fluid systems and generally arise from pressure driven flow and viscous properties of the fluid. In general, the surface forces are expressed as the divergence of the stress tensor:

\begin{equation}
    \textbf{f}_s = \boldsymbol{\nabla} \cdot \Tau
\end{equation}
\begin{equation}
    \textbf{f}_s = \frac{d\tau_{ij}}{dx_i}
\end{equation}

\noindent where $\textbf{f}_s$ is the surface force, and $\Tau$ and $\tau_{ij}$ are the second order stress tensor which is dependent on the type of fluid and the driving forces. For pressure driven inviscid fluids, the stress tensor and surface forces can be expressed as 

\begin{equation}
    \tau_{ij} = -p\delta_{ij}
\end{equation}
\begin{equation}
    \textbf{f}_s = - \frac{dp}{dx_j}
\end{equation}

\noindent where p is the pressure. 

\par For a viscid Newtonian fluid, the stress tensor and surface force can be expressed as

\begin{equation}
    \tau_{ij} = -p\delta_{ij} + 2\mu(\epsilon_{ij} - \frac{1}{3}\epsilon_{kk}\delta_{ij})
\end{equation}
\begin{equation}
    \textbf{f}_s = \frac{dp}{dx_i} + \mu \frac{d}{dx_j}(\frac{du_i}{dx_j} + \frac{du_j}{dx_i} - \frac{2}{3}\delta_{ij}\frac{du_k}{dx_k})
    \label{eqn:viscid_newt_force}
\end{equation}

\noindent where $\epsilon_{ij}$ is the rate of strain tensor written as 

\begin{equation}
    \epsilon_{ij} = \frac{1}{2} \Big(\frac{du_i}{dx_j} + \frac{du_j}{du_i}\Big).
\end{equation}

\noindent If the fluid is incompressible, then conservation of mass states

\begin{equation}
    \frac{du_k}{dx_k} = \boldsymbol{\nabla} \cdot \textbf{u} = 0,
\end{equation}

\noindent and equation \ref{eqn:viscid_newt_force} can be simplified to 

\begin{equation}
    \textbf{f}_s = -\frac{dp}{dx_i} + \mu \frac{d^2u_i}{dx^2_j}
\end{equation}
\begin{equation}
    \textbf{f}_s = -\boldsymbol{\nabla}p + \mu\boldsymbol{\nabla}^2\textbf{u} 
\end{equation}

\par Additional common external forces on the control volume can include the gravitational force

\begin{equation}
    \textbf{f}_g = \rho \textbf{g},
\end{equation}

\noindent where \textbf{g} is the acceleration of gravity, and for some microfluidic applications, the electroosmotic flow force (EOF)

\begin{equation}
    \textbf{f}_{EOF} = -\rho_e \boldsymbol{\nabla}\phi,
\end{equation}

\noindent where $\phi$ is the applied electric potential, and $\rho_e$ is the net charge density of the electric double layer (EDL). The electroosmotic force arises from a net charge that appears along the surface of channels under an applied voltage known as the electric double layer. This voltage creates 

\subsection*{Laminar Flow}
\par Laminar flow describes the condition where the velocity of a particle in fluid flow is not a random function of time in contrast to turbulent flow which is chaotic \cite{pamb}. The Reynold number can quantitatively characterize a fluid flow. The Reynold number is a dimensionless number that is a ratio of inertial and viscous forces and can be expressed as

\begin{equation}
    \text{Re} = \frac{\rho v L}{\mu},
\end{equation}

\noindent where $\rho$ is the fluid density, $v$ is the characteristic fluid velocity, $\mu$ is the fluid viscosity, and $L$ is the characteristic length. In many cases, the characteristic length is the hydraulic diameter ($D_h$). The hydraulic diameter is used for fluid calculations in non-circular conduits by relating the conduit to a circular geometry in a proportion that maintains the conservation of momentum of the original conduit \cite{PAMB}.

\par The hydraulic diameter can be expressed as ratio of cross-sectional area to conduit perimeter:

\begin{equation}
    D_h = 4 \frac{A}{P},
    \label{eqn:hydraulic_diameter}
\end{equation}

\noindent where $A$ and $P$ is cross-sectional area and perimeter of the conduit respectively. For a cylindrical conduit, equation \ref{eqn:hydraulic_diameter} simplifies to $D_h = D_c$ where $D_c$ is the diameter of circular cross-section. For a rectangular cross-section, the hydraulic diameter is expressed as 

\begin{equation}
    D_h = \frac{2hw}{w+h},
\end{equation}

\noindent where $h$ and $w$ are the height and width of the cross-sectional rectangle respectively. For turbulent flows, where the geometry of lesser consequence, equation \ref{eqn:hydraulic_diameter} is a good approximation for fluid calculations, but for laminar flow specifics of the conduit geometry is of great consequence and equation \ref{eqn:hydraulic_diameter} should be used with caution and the understanding that results may be inaccurate. 

\par In general, flow conditions with a Reynolds number much larger than 2300 exhibit turbulent behaviours, and flow conditions with a Reynolds number much smaller than 2300 exhibit laminar flow behaviours. The laminar-turbulent transition number of 2300 is reportedly accurate for microfluidics as well \cite{PAMB-14}.

\par Referring back to equation \ref{eqn:hydraulic_diameter}, since the product of characteristic length and velocity for microfluidic systems, most microfluidic flows are considered laminar. An important consequence is that separate streams that come in contact will not mix via convection, but only through diffusion. This phenomenom can be quantified with the dimensionless P\'eclet number (Pe), which is defined as the ratio of the convection transport to diffusion transport \cite{micromixers}. The P\'eclet number can be expressed as 

\begin{equation}
    \text{Pe} = \frac{vL}{D}
\end{equation}

\noindent where D is the diffusion coefficient. The P\'eclet number can also be defined with the Reynolds number

\begin{equation}
    Pe = \text{Re}\;\text{Sc}
\end{equation}

\noindent where Sc is the dimensionless Schmidt number and describes the ratio of momentum diffusivity and and mass diffusivity. The Schmidt can be calculated as

\begin{equation}
    \text{Sc} = \frac{\mu}{\rho D}
\end{equation}

\par For fluid systems where Pe is much larger than 1, the system is convection dominated and when Pe is much smaller than 1, the system is diffusion dominated. Again, the product of the characteristic velocity and length in microfluidic systems is usually very small so in most BioMEMS, the system is diffusion dominated and cannot rely on convective mixing. To facilitate mixing in microfluidic systems, the device should be designed to create large surface areas between streams that must be mixed to expedite the diffusion process or stimulate turbulent flow to create convective transport.

\subsection*{Diffusion}

\begin{figure}[ht]
    \centering
    \includegraphics[width=\textwidth]{images/diffusion_illustration.png}
    \caption{Illustration of diffusion}
    \label{fig:diffusion_ilustration}
\end{figure} 



 %%%%%%%%%%%%%%%%%%%%%%%%%%%%%%%%%%%%%%%%%%%%%%%%%%%%%%%%%%%%%%%%%%%
 % Previous Work on the Cal Poly Biofluidic Lab's EIS System
 %%%%%%%%%%%%%%%%%%%%%%%%%%%%%%%%%%%%%%%%%%%%%%%%%%%%%%%%%%%%%%%%%%%
 \section{Previous Work on the Cal Poly Biofluidic Lab's EIS System}
 
 % In 2009 Josh Fadriuela and Stephanie Hernandez fulfilled their thesis under Dr.Clague to create a cell impedance sensor system. Their work will be the foundation for this thesis \cite{fadriquela_design_2009-1}, \cite{hernandez_single_2009-1}.
 