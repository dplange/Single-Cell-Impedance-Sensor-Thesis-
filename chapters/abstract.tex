
\par Impedance spectroscopy (IS) is an important tool for cell detection and characterization in medical and food safety applications. In this thesis, the Cal Poly Biofluidics Lab's impedance spectroscopy system was re-evaluated and optimized for single-cell impedance spectroscopy. To evaluate the IS system, an impedance spectroscopy bioMEMS chip was fabricated in the Cal Poly Microfabcrication lab, software was developed to run IS experiments, and studies were run to validate the system. To explore IS optimization, Maxwell's mixture theorem and the Schwartz-Christoffel transform was used to calculate an analytic impedance solution to the co-planar electrode system, a novel volume fraction to account for the non-uniformity of the electric field was developed to increase the accuracy of the analytic solution and to investigate the effect of cell position on the impedance spectrum, a software program was created to allow easy access to the analytic solution, and FEA models were developed to compare to the analytic solution and to investigate the effect of complex device geometry.



