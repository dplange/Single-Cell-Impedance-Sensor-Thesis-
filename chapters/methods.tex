
\section{Device Manufacturing}

\subsection{PDMS Channel Fabrication}
\par The device microchannels were fabricated by pouring PDMS (polydimethysiloxane) over a silicon wafer master mold. Through photolithography, a master mold was created by patterning a silicon wafer with SU-8 photoresist in  a process known as soft lithography. A 6"x6" 20,000 DPI negative transparency mask was ordered from CAD/Art Services Inc. with the emulsion face down. A general process flow is depicted in figure \ref{fig:soft_lithography}.


\subsection*{SU-8 Master Mold}

\par Before photolithography, silicon wafers were cleaned in a Piranha bath ( 98\% sulfuric acid, get pirahna def) for 15 minutes and then rinsed in DI (deionized) water, and dipped in BOE solution (get BOE def) for 5 minutes and then rinsed in DI water. The wafers were then washed and dried in the SRD (spin rinse dry) machine before baking the wafers at 205$^\circ$ for 10 minutes on a hot plate to dehydrate the wafers, and allowed to cool for 10 minutes. 

\begin{figure}[h]
    \centering
    \includegraphics[width=0.7\textwidth]{images/resist_spinner_open.jpg}
    \caption{Laurel Technologiers ws-400 spin coater}
    \label{fig:spin_coater}
\end{figure}

\par The wafers were coated with SU-8 (a negative tone photoresist) on a spin coater (Laurel Technologies, ws-400; figure \ref{fig:spin_coater}). The spin coater rotates at a series of specific angular velocities in order to coat the wafer with the desired thickness of photoresist. About 4 mL of SU-8 2007 (\#07110769, MicroChem) was placed on the center of the wafer and then spun for 400 RPM for 20 seconds to disperse the SU-8, and then at 1500 RPM for 35 seconds to spread the photoresist to a 10 $\mu$m thickness. After spin-coating, the wafer was soft-baked at 85$^\circ$C for 3 minutes and allowed to cool for 4 minutes.

\begin{figure}[h]
    \centering
    \includegraphics[width=0.7\textwidth]{images/aligner.jpg}
    \caption{Cannon PLA501FA mask aligner}
    \label{fig:mask_aligner}
\end{figure}

\par The photolithography process was completed using the Cannon PLA-501FA mask aligner (figure \ref{fig:mask_aligner}). With the transparency mask centered over the wafer and the 365 nm glass-transparency filter in place, UV light was applied to the photoresist for 14 seconds with a total exposure of 140 mJ/cm$^2$. The wafer was then baked at 85 $^\circ$C for 4 minutes with a 10 minute cool down. 

\par The wafer was then developed in propylene glycol monomethyl ether acetate (SU-8 developer, MicroChem) in order to remove the non-exposed SU-8 from the wafer. The wafer was placed in the SU-8 developer for 3 minutes. The wafer was then hard baked for 15 minutes at 210 $^\circ$C. 

The detailed procedure steps are provided in appendix \ref{app: su-8_photolith}.

\subsection*{Soft Lithography}
\par The polydimethylsiloxane (PDMS) for soft lithography was obtained by purchasing a Dow Corning 184 Sylgard kit through Ellsworth adhesives. The kit comes with a base and a curing agent that cross links the PDMS and increases the polymer's stiffness. The PDMS was prepared by mixing the base and the curing agent with a 10:1 ration. The mixture was then degassed under vacuum until the PDMS was clear and free of bubbles (figure \ref{fig:pdms_vacuum}). The PDMS was then poured directly onto the SU-8 master mold until the PDMS covered the wafer with approximately 1/4" depth. The PDMS was cured by baking in an oven for at least an hour at 65 $^\circ$. The PDMS chips were cut from the master mold with a scapel and peeled off the wafer. 

\begin{figure}[h]
    \centering
    \includegraphics[width=0.65\textwidth]{images/pdms_in_vacuum.jpg}
    \caption{Degassing PDMS mixture under vacuum.}
    \label{fig:pdms_vacuum}
\end{figure}


\par The finished product was a PDMS chip with the desired microchannel dimensions molded into the polymer surface. The procedure steps for soft lithography are provided in appendix \ref{app: soft_litho}.


\subsection{Electrode Fabrication}

\par The device electrodes were fabricated onto a glass substrate using the lift-off process (section \ref{sec: lift_off}). The process utilized a 6"x6", 20,000 DPI transparency mask ordered from CAD/art services Inc. A flow chart of the overall process is depicted in figure \ref{fig:lift_off}.

\subsection*{Photolithography}

\par Glass wafers were prepared for the lift-off process by cleaning the wafers in a pirahna bath (pirahna def??) for 15 minutes and rinsing in DI water, and then a 1 minute dip in BOE (BOE definition???) and rinsing in DI before running the wafers through the SRD (SRD definition???). The wafers were then dehydrated by baking on a hot plate for 10 minutes at 200 $^\circ$C and then allowed to cool down fro five minutes. 

\begin{figure}[h]
    \centering
    \includegraphics[width=0.8\textwidth]{images/resist_spinner_open.jpg}
    \caption{Laurel technologies ws-400 spin coater}
    \label{fig:spin_coater}
\end{figure}

\par The wafers were coated with ma-N1420 negative-tone photoresist on a spin coater (Laurel technologies, ws-400; \ref{fig:spin_coater}). The ma-N1420 photoresist is advantageous for the lift-off process since the developed photoresist has an undercut profile and the cross-linked photoresist is durable in contrast to the "soft" positive photoresist. Before application of the ma-N1420 a primer (primer definition???) is dispersed and then spun onto the wafer at 3000 RPM for 30??? seconds. Ma-N1420 is then dispersed onto the glass wafer, but before spinning, ensure that the photoresist spreads to all edges of the wafer. The photoresist was spun at 2000 RPM for 35??? seconds with a target of 2.5 micron thickness. The wafers were then soft-baked on a hotplate at 120 $^\circ$C for 3 minutes to increase stability. 

\par The photolithography process was completed using the Cannon PLA-501FA mask aligner (figure \ref{fig:mask_aligner}). The glass wafer was placed in the aligner with a black tape backing in order to prevent light scattering and absorb radiation, and the transparency was centered over the wafer. UV light was applied to the wafer for 30 seconds for a total of 450 mJ/cm$^2$. 

\par The wafer was then developed for 60-90 seconds in something?? based developer (ma-D 533/3) and then rinsed in DI water. At this step, it is critical to examine the wafer under a microscope to ensure the photoresist is entirely removed from the substrate where the electrode is desired and that the electrode edges are sharp. If not, develop for another 10 seconds and reexamine. 

\par After the photoresist is fully developed, the wafer was flood-exposed (i.e. no transparency) in order to improve stability. The photoresist was exposed 3 times for 40 seconds with a five minute rest between each exposure for a total of 1800 mJ/cm$^2$ of exposure. Finally, the wafers were baked on the hotplate with a ramping temperature from 60 $^\circ$C to 100 $^\circ$C for 10 minutes, and then allowed to cool for 10 minutes. 

\subsection*{Metal Deposition}

\par To deposit the electrode metals onto the developed-photoresist wafer, the AGS reactive ion etcher, CrC-150 chrome sputterer, and the Denton Desk V sputtering system were used to clean and sputter chromium and gold. The target metal deposition consists of three layers: a 125 \si{\angstrom} thick base chromium layer to facilitate adhesion of the electrode to glass, a 180 nm thick gold layer as the main conductor in the electrode, and a 42 \si{\angstrom} thick layer of chromium that will oxidize to an inert finish (figure \ref{fig:metal_deposition}).

\begin{figure}[h]
    \centering
    \includegraphics[width=\textwidth]{images/metal_deposition_diagram.png}
    \caption[Diagram of metal deposition onto photoresist-developed substrate]{Diagram of the three layered metal deposition onto photoresist-developed substrate.}
    \label{fig:metal_deposition}
\end{figure}

\par To prepare the photoresist-developed wafer for metal deposition, the wafers were exposed to an oxygen plasma to volatilize and remove organic residues. The wafer was placed under vacuum in the AGS RIE system and exposed to an oxygen plasma for 30 seconds (figure \ref{fig:AGS-RIE}). The CrC-150 sputtering system was then used to deposit 125 \si{\angstrom} of chromium at a rate of 250 \si{\angstrom} per minute for 30 seconds. Using the Denton Desk 5, 180 nm of gold was deposited at a rate of 180 \si{\angstrom} per minute for 10 minutes. A final 42 \si{\angstrom} layer of chromium was deposited with the CrC-150 with a rate of 250 \si{\angstrom} per minute for 10 seconds (figure \ref{fig:sputterers}). 

\begin{figure}
    \centering
    \includegraphics[width=0.7\textwidth]{images/AGS_RIE.jpg}
    \caption[AGS reactive ion etcher]{The AGS reactive ion etcher used to remove organic residue with an oxygen plasma.}
    \label{fig:AGS-RIE}
\end{figure}

  \begin{figure}[h]
    \centering
    \begin{subfigure}[b]{0.45\textwidth}
        \centering
        \includegraphics[width=\textwidth]{images/CrC-150.jpg}
        \caption{CCr-150 Sputter system}
        \label{fig:ccr-150}
    \end{subfigure}
    \hfill
    \begin{subfigure}[b]{0.45\textwidth}
        \centering
        \includegraphics[width=\textwidth]{images/denton.jpg}
        \caption{Denton Desk V Sputter/Etch unit}
        \label{fig:denton}
    \end{subfigure} 
    \caption[Sputter machines used for metal deposition]{The sputter machines used for metal deposition. The CCr-150 and the Denton Desk V were used for chromium and gold deposition respectively.}
    \label{fig:sputterers}
 \end{figure}

\par During the transition between sputtering systems, the wafers were kept under vacuum in order to keep them clean, and in the case for chromium depositions, to reduce the formation of chromium oxide which can significantly impair chromium-gold adhesion.

\subsection*{Lift Off}

The final step of the electrode fabrication process was to remove the photoresist with the undesired metal depositions. The wafers were submerged in Microposit remover 1165 for five days under constant agitation (figure \ref{fig:lift_solution}). To increase the speed of lift off, the solution can be heated to 65$^\circ$C.

\begin{figure}
    \centering
    \begin{subfigure}[b]{0.45\textwidth}
        \includegraphics[width=\textwidth]{images/lift_off.jpg}
        \caption{Sputtered wafer in Microposit remover}
        \label{fig:lift_solution}
    \end{subfigure}
    \hfill
    \begin{subfigure}[b]{0.45\textwidth}
        \centering
        \includegraphics[width=\textwidth]{images/electrodes.jpg}
        \caption{Wafer with electrode post lift off}
        \label{fig:electrode_methods}
    \end{subfigure} 
    \caption[Lift off]{The lift off process with metal depositions removed from the glass substrate with Microposit remover 1165 in (a) and the remaining depositions after the lift off process in (b).}
    \label{fig:electrode_methods}
\end{figure}

\subsection{Device Assembly: Alignment and Plasma Bonding}

\par To assemble the PDMS chip containing the microfluidic channels to the electrodes on the glass substrate, the components were plasma bonded to create a water-tight seal that permits optical viewing. The PDMS and the galss electrode substrate were placed device-side up in the Cal Poly Microfabrication Lab's PDC-32G plasma cleaner. After pumping the plasma cleaner down to a vacuum, the device was exposed to an atmospheric plasma for 10 seconds.  

\begin{figure}
    \centering
    \includegraphics[width=\textwidth]{images/plasma_cleaner.jpg}
    \caption[PDC-32G plasma cleaner]{PDC-32G plasma cleaner used for plasma bonding.}
    \label{fig:my_label}
\end{figure}

\par After plasma activation, a drop of ethanol was place don the PDMS surface before placing the PDMS onto the electrode substrate in order to create a liquid float barrier to that will prevent permanent binding for $\sim$5 minutes. The electrodes and PDMS microchannels were aligned by hand with a microscope set at 20X magnification for visual feedback. After proper alignment, the device was baked in an oven at 65$^\circ$C for an hour. 



\section[Software]{Software Interface}

\par LabVIEW was used to interface with the NI-5421 function generator and the NI-5124 oscilloscope to package the circuit model, data acquisition, and data analysis into an impedance spectroscope program. The LabVIEW language was chosen due to the ease of connecting to the National Instrument hardware and the rapid development cycle. The program specifications include interfacing with the NI-hardware to run an impedance spectroscopy experiment, the ability to control the sweeping frequencies, the ability to switch circuit topologies, the ability to automatically run tests at specified intervals, the implementation of a data saving system that records the users settings for the experiment as well as the impedance spectroscopy results, and the ability to view and compare data.

\begin{figure}[h]
    \centering
    \begin{subfigure}[b]{\textwidth}
        \centering
        \includegraphics[width=0.85\textwidth]{images/program_overview.png}
        \caption{Event-driven state machine architecture}
        \label{fig:IS_architecture}
    \end{subfigure}
    \vspace{0.1 in}
    \vfill
    \begin{subfigure}[b]{\textwidth}
        \centering
        \includegraphics[width=\textwidth]{images/event_driven_SM.png}
        \caption{Event-driven state machine code setup}
        \label{fig:IS-base_code}
    \end{subfigure}
    \caption[Event-driven state machine architecture]{The event-driven state machine architecture used for the impedance spectroscope LabVIEW program. Events override the state of the state machine that runs inside the time out event.}
    \label{fig:IS_software_design}
\end{figure}
\FloatBarrier

\subsection{Architecture}

\par The LabVIEW program was designed with an event-driven state machine architecture and is diagrammed in figure \ref{fig:IS_software_design}. A state machine is a common LabVIEW coding pattern where the program exists as a set of states. Depending on user input or calculations, each state leads to a subsequent state that can potentially lead to a large network of modular decision making states. For the impedance spectroscope, the state machine is simple with most states leading to an idle state that continues to wait for user input. The state machine lives inside an event structure that listens for specific user interface (UI) interactions that triggers an event that runs code and can specify the next state in the state machine. After a specified amount of time passes without an event triggering, the time out event will trigger and run the state machine.    

\par The event driven state machine architecture will allow for a responsive UI that can adapt outside of the state machine architecture while maintaining the modularity of a state machine with safe program initialization and exit. The implementation of the software resulted in two main functions: impedance spectroscopy, and the data viewer.

\subsection{Impedance Spectroscopy}
\par The impedance spectroscopy UI gives users the ability to set settings for experiments and view the results (figure \ref{fig:is_gui}). All of the gui settings are located in a set of three tabs: circuit topology, frequency parameters, and interval parameters. 

\begin{figure}[h]
    \centering
    \includegraphics[width=\textwidth]{images/IS_gui.png}
    \caption{Impedance spectroscopy user interface}
    \label{fig:is_gui}
\end{figure}

\par The circuit topology tab tell the software what circuit is being used and the value of the external resistor (figure \ref{fig:labview_circuit_settings}). With each circuit selection, an image below updates to display where the software expects the oscilloscope channels to be connected and the formula calculating impedance. In the current implementation, the software includes options for the I-V and auto-balancing bridge circuits. 


\begin{figure}[h]
    \centering
    \begin{subfigure}[b]{0.48\textwidth}
        \centering
        \includegraphics[width=\textwidth]{images/labview_circuit_IV.png}
        \caption{I-V topology}
        \label{fig:labview_I-V_circ}
    \end{subfigure}
    \hfill
    \begin{subfigure}[b]{0.48\textwidth}
        \centering
        \includegraphics[width=\textwidth]{images/labview_circuit_auto.png}
        \caption{Auto-balancing bridge topology}
        \label{fig:labview_auto_circ}
    \end{subfigure}
    \caption{Circuit settings}
    \label{fig:labview_circuit_settings}
\end{figure}

\par The frequency parameter tab contains the controls that determine what frequencies are swept over in the impedance spectroscopy experiment (figure \ref{fig:labview_frequency_settings}). The frequency expression shows how frequency values are calculated with x and y representing incremented values with the caveat that a value for every increment of x is generated for every one increment of y. For example, with the settings given in figure \ref{fig:labview_frequency_settings}, ten frequencies per decade with decades 1E3 to 1E7 are generated. The max input frequency truncates the frequencies generated to the specified value. In addition to frequency parameters, the tab also includes controls for the waveform amplitude, the number of cycles measured by the oscilloscope and the sample rate. 

\begin{figure}[h]
\centering
\begin{minipage}[t]{.48\textwidth}
  \centering
  \includegraphics[width=\textwidth]{images/labview_freq.png}
  \captionof{figure}{Frequency sweep options}
  \label{fig:labview_frequency_settings}
\end{minipage}
\vspace{0.2 in}
\vfill
\begin{minipage}[t]{.48\textwidth}
  \centering
  \includegraphics[width=\textwidth]{images/labview_intervals.png}
  \captionof{figure}{Interval measurement settings}
  \label{fig:labview_interval_settings}
\end{minipage}
\end{figure}

\par The interval parameters tab contains options for running automatic experiments at timed intervals (figure \ref{fig:labview_interval_settings}). 

\begin{figure}[h]
    \centering
    \begin{subfigure}[b]{0.48\textwidth}
        \centering
        \includegraphics[width=\textwidth]{images/labview_mag_phase_graph.png}
        \caption{Impedance magnitude and phase}
        \label{fig:labview_mag-phase_graph}
    \end{subfigure}
    \hfill
    \begin{subfigure}[b]{0.48\textwidth}
        \centering
        \includegraphics[width=\textwidth]{images/labview_real_imag_graph.png}
        \caption{Real and imaginary impedance graphs}
        \label{fig:labview_real-imag_graph}
    \end{subfigure}
    \vspace{0.2in}
    \vfill
    \begin{subfigure}[b]{0.48\textwidth}
        \centering
        \includegraphics[width=\textwidth]{images/labview_nyquist_graph.png}
        \caption{Nyquist plot}
        \label{fig:labview_nyquist_plot}
    \end{subfigure}
    \hfill
    \begin{subfigure}[b]{0.48\textwidth}
        \centering
        \includegraphics[width=\textwidth]{images/labview_bode_phase_graph.png}
        \caption{Bode pot and phase shift plot}
        \label{fig:labview_bode-phase_plot}
    \end{subfigure}
    \caption{Impedance spectroscopy graphs}
    \label{fig:labview_IS_graphs}
\end{figure}

\par After running an experiment, the results are visualized and then saved in a text file. The results are used to generate seven plot in four different tabs: impedance magnitude and phase, real and imaginary impedance, Nyquist plot, and a voltage bode and phase plots (figure \ref{fig:labview_IS_graphs}. With the exception of the voltage bode and phase plots, the plots are different ways of visualizing the same impedance data. The voltage bode and phase plots were primarily used in troubleshooting. 


\begin{figure}[h]
    \centering
    \includegraphics[width=0.8\textwidth]{images/labview_data.png}
    \caption{Example of impedance spectroscopy data file}
    \label{fig:labview_data}
\end{figure}

\par Every experiment is automatically saved in a tab delimited text file with the designated experiment name (a number is appended to a file name duplicates; i.e. no files are overwritten) with the option to generate dated folders. Each file contains a header that saves the date and time the experiment was run, the user, and user defined settings. The impedance spectroscope data is saved in labeled columns. An example of a saved experiment file is given in figure \ref{fig:labview_data}.
\FloatBarrier

\subsection{Data Viewer}

\par The data viewer UI allowed users to visualize and compare previous experiments (figure \ref{fig:labview_data_viewer_gui}). The UI allows up to ten experiments to be simultaneously plotted with a legend in the top right corner and a table displaying each experiment's header information.

\begin{figure}[h]
    \centering
    \includegraphics[width=\textwidth]{images/labview_dataViewer.png}
    \caption{Data viewer user interface}
    \label{fig:labview_data_viewer_gui}
\end{figure}

\section{System Implementation}
\par To run impedance spectroscopy tests, a system was assembled to control and record experiments. The complete implementation included a function generator (NI PXI-5421), an oscilloscope (NI PXI-5124), a hardware interfacing labview program, an I-V circuit, three Harvard apparatus syringe pumps, an inverted video microscope (LabSmith SVM340) to optical feedback, and the impedance spectroscopy microfluidic chip. Figure \ref{fig:IS_system} displays the impedance spectroscopy system. 

\begin{figure}[h]
    \centering
    \includegraphics[width=\textwidth]{images/impedance_system}
    \caption{The impedance spectroscopy system}
    \label{fig:IS_system}
\end{figure}

\subsection{Electrical Interface}

\subsection*{NI Hardware}
\par The NI PXI-5421 function generator and the NI PXI-5124 oscilloscope were used to load and measure the impedance spectroscopy chip. The NI PXI-5421 is a 43 MHz waveform generator capable of generating user defined standard and arbitrary waveforms with a $\pm$6 V range and a 50 $\Omega$ output impedance. The signal generator was controlled with the LabVIEW program and operated to run at specified frequencies of a standard sine wave. The NI PXI-5124 is a 150 MHz 200 MS/s oscilloscope. The PXI-5124 has a 4.0 GS/s equivalent time sampling for repetitive signals, and a selectable 50 $\Omega$ or 1 M$\Omega$||29 pF input impedance. The oscilliscope was controlled with LabVIEW and configured to have a 1 M$\Omega$||29 pF input impedance.

\subsection*{Electrode Interface}
\par To establish an electrical connection to the impedance spectroscopy chip, solid core wires were soldered to to the device. Since the metal depositions were too thin for direct soldering, copper tape was placed adjacent to the soldered pads, soldered to the solid-core wire, and then a silver-based epoxy was applied to the copper tape electrode pads interface using the M6 Chemicals silver conductive pen. The epoxy has a resistivity of 1E-4 $\Omega\cdot$cm.

\par Connections to the NI instruments were made with solid-core wires soldered to BNC female jacks. Since the function generator output signal was not 50 $\Omega$ terminated, the signal amplitude will be twice the selected amplitude. Wiring length to the I-V circuit was minimized. 

\subsection*{I-V Circuit}
\par An I-V circuit, with an external resistor of 3300 $\Omega$, measured the impedance of the device under testing. The circuit was discussed in detail in section \ref{sec:impedance_spectroscopy}, and the implementation is depicted in figure \ref{fig:I-V_implementation}.

\par The circuit was implemented using two different methods: a system of soldered solid core wires, and using a breadboard. There was no difference noted between the two implementations sweeping up to 40 MHz. However, the breadboard should generally be avoided for high frequencies due to intercontact capacitance ($\sim$25 pF for power strips, and $\sim$2.5 pF for the remaining breadboard).


\subsection{Microfluidics}

\par Three Harvard Apparatus syringe pumps driving 1 mL BD syringes were used to control fluid flow through the impedance spectroscope chip. The microfluidic network is diagrammed in figure \ref{fig:ufluidic_network}.

\par The original design of the device was meant to capture a single cell in the measurement, but since this design cannont achieve single-cell isolation (section \ref{ch: introduction}), pump \#3 was used to initially flood the flush channels and then hold its syringe at a constant volume to prevent flow through the flush channels. For fluid flow, pump \#1 drove fluid at 1 $\mu$L/min and pump \#2 pulled fluid at 1 $\mu$L/min. 

%The flow with water was calculated to generate an estimated pressure difference of \#\# between the inlet and outlet, and a max gauge pressure of \#\# (appendix \ref{fluid_flow_pressure}).

\par Spherotech polystyrene beads (FP-2052-2) were injected into the impedance spectroscope system for proof of concept. The Spherotech beads ranged from 7 $\mu$m to 7.9 $\mu$m in diameter and had a relative permittivity of 2.5 (table \ref{tab:permittivity_table}). The beads were suspended in a DI solution by 0.1\% by mass with a single drop of Titron x-100 surfactant and then sonicated.

\par The device was evaluated by making impedance measurements of DI water, 1x phosphate buffered saline solution (PBS), DI with beads, and PBS with beads. 

