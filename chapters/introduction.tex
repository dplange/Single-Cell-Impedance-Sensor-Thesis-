\par Cell detection and characterization are important for applications in medical diagnosis and food safety \cite{mansor}. With the advent of micro-electro-mechanical systems (MEMS), cell characterization techniques have developed with increasing precision and cost-effectiveness. One of these techniques is impedance spectroscopy. Cell impedance spectroscopy measures the electrical impedance of cell(s) over a range of frequencies and can identify cell types, differentiate cell states, and provide information about cell components. With significantly increased manufacturing precision provided by MEMS technologies, cell impedance spectroscopy has been miniaturized to measure the impedance spectrum of single cells instead of macro suspensions. 

\par The applications of single cell impedance spectroscopy are extensive. In medical diagnostics, the ability to isolate the impedance spectrum of a single cell allows the diagnosis of diseases with very small limits of detection, such as early detection of cancer by identifying circulating tumor cells that can be as scarce as 1 cell per mL of blood \cite{mansor-1}. In food safety single cell spectroscopy can detect pathogens in a rapid and inexpensive manner, such as E.Coli and Salmonella contamination in water sources \cite{mansor-3}. In addition, impedance spectroscopy is an important tool in research, with the ability to quickly measure a cell's state and response to stimuli. 

\par The focus of this thesis will be to reevaluate and optimize the Cal Poly Biofluidic lab's single cell impedance sensor.

\section[Cal Poly's IS System]{The Cal Poly Biofluidic Lab's Cell Impedance System}

\par In 2009 Josh Fadriquela and Stephanie Hernandez created an cell impedance spectroscope system with the intention o

\section{Thesis Purpose and Objectives}